\capitulo{1}{Introducción}

El sector de la restauración es un entorno dinámico y complejo donde la eficiencia operativa es un factor clave para el éxito. Tradicionalmente, la gestión de estos establecimientos ha dependido de una combinación de herramientas dispersas (desde registros manuales hasta software especializado a menudo cerrado y poco adaptable) que no siempre se comunican entre sí. Esta fragmentación de procesos aumenta la complejidad operativa y el margen de error, dificultando el flujo de trabajo entre cocina y sala.

En este contexto, \textit{Orderly} surge como un proyecto con el objetivo de desarrollar una solución web integral, moderna y centralizada. La aplicación busca ser un \textbf{punto de gestión unificado} que cubra los procesos esenciales de un local: desde la creación y composición de platos, pasando por el registro de comandas y la gestión de reservas, hasta la administración de los roles del equipo. Al integrar estos módulos en una plataforma modular y extensible, se pretende no solo agilizar los flujos de trabajo, sino ofrecer una herramienta que se adapte con facilidad a distintos tipos de establecimientos y necesidades.

Un pilar diferenciador de este proyecto es su compromiso con la \textbf{transparencia nutricional}. La integración directa con la API de \textit{\textbf{Open Food Facts}} permite a los establecimientos automatizar la consulta de ingredientes y alérgenos, facilitando la elaboración de fichas de productos precisas y verificadas. En una era donde la conciencia sobre la alimentación saludable es creciente, esta funcionalidad añade un valor significativo y contemporáneo a la herramienta.

Este documento constituye la \textbf{memoria} principal del proyecto, donde se detallan los objetivos, conceptos teóricos, técnicas empleadas y los aspectos más relevantes del desarrollo. La documentación técnica completa, que incluye la planificación detallada del proyecto, la especificación de requisitos y diseño, así como los manuales de programador y usuario, se presenta en un conjunto de \textbf{anexos} independientes.

Todo el código fuente, la documentación y el histórico de desarrollo están disponibles públicamente en el repositorio de \textit{GitHub} del proyecto:

{\centering \url{https://github.com/aperezolmos/Orderly} \par}

El progreso del proyecto ha sido gestionado mediante la herramienta \textit{GitHub Projects}, cuyo tablero y gráficas asociadas pueden visualizarse en:

{\centering \url{https://github.com/users/aperezolmos/projects/4} \par}

El reporte de calidad del código realizado por \textit{SonarCloud} puede visualizarse en:\todo{Aún está privado}

{\centering \url{https://sonarcloud.io/project/overview?id=aperezolmos_Orderly} \par}
