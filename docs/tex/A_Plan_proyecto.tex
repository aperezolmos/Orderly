\apendice{Plan de Proyecto Software}

\section{Introducción}

\todo[inline]{Pendiente escribir.}

%%%%%%%%%%%%%%%%%%%%%%%%%%%%%%%%%%%%%%%%%%%%%%%%%%%%%%%%%%%%%%%%

\section{Planificación temporal}

\todo[inline]{Pendiente escribir.}

%%%%%%%%%%%%%%%%%%%%%%%%%%%%%%%%%%%%%%%%%%%%%%%%%%%%%%%%%%%%%%%%

\subsection{\textit{Sprint} 0}
\label{subsec:A.2_sprint0}

\underline{Duración}: 1 semana (18/09/2025 - 25/09/2025)

Este primer \textit{sprint} abarca las tareas de investigación y definición inicial, enfocadas en determinar el alcance mínimo del proyecto, junto con las tecnologías y recursos a utilizar.

Los \textit{issues} creados para cumplir los objetivos definidos se pueden ver en: \href{https://github.com/aperezolmos/tfg-aperezolmos/milestone/1?closed=1}{\textbf{Sprint 0}}

\subsubsection{Objetivos}

\begin{itemize}
	\item \textbf{Creación del repositorio:} Se configuró el repositorio de \textit{GitHub} para el control de versiones. Utilizando \textit{GitHub Projects}, se asoció un proyecto a dicho repositorio para la gestión y mantenimiento de las tareas de cada \textit{sprint}.
	\item \textbf{Creación de los \textit{epics}\todo{ref de qué son¿?}:} Se definieron \textit{issues} principales para estructurar el proyecto en áreas funcionales (desarrollo, \textit{testing}, documentación) y agrupar tareas relacionadas, agilizando el acceso y el seguimiento del progreso.
	\item \textbf{Definición del \textit{stack} tecnológico:} Se eligió \textit{\textbf{Spring Boot}} (con \textit{Spring Security} y \textit{Spring Data JPA}) para el \textit{backend} y la lógica de negocio, junto con \textit{\textbf{MySQL}} como base de datos. Para el \textit{frontend}, se optó por \textit{\textbf{Thymeleaf}}, que facilita la generación de vistas HTML dinámicas desde el servidor. Finalmente, se seleccionó \textit{\textbf{Docker}} para los contenedores de la aplicación y la base de datos, asegurando un despliegue portable y reproducible.
	\item \textbf{Estudio y evaluación de APIs nutricionales:} Se realizó una comparativa entre varias APIs de alimentos. La API elegida fue \textit{Nutritionix}\todo{ref?}. Los criterios de elección y el resto de candidatas pueden verse en el \textit{issue} \href{https://github.com/aperezolmos/tfg-aperezolmos/issues/10}{\#10}.
	\item \textbf{Revisión de trabajos previos y TFGs relacionados:} ...NutriMenu...\todo{referencia bibliografía}
	\todo[inline]{Pendiente completar.}
	\item \textbf{Análisis de integración con plataformas de \textit{delivery}:} Se investigó acerca de los requisitos necesarios para obtener claves de acceso a las APIs de distintas plataformas de \textit{delivery}. Se terminó por concluir que esta integración se pospondría para etapas más avanzadas del desarrollo, en caso de que se dispusiera del tiempo necesario.
\end{itemize}

\imagenConEtiqueta{img/apen_A/A2_sprint0_burndown.png}{Gráfico \textit{burndown} del Sprint 0}{A2_sprint0_burndown}

\subsubsection{Revisión}

Como la API de \textit{Nutritionix} no proporciona directamente datos de alérgenos, se valoró la idea de \textbf{``\textit{mapear}'' los alérgenos} en base a los ingredientes, mediante un modelo de reglas y una lista predefinida de alérgenos. Se planteó la posibilidad de que existiese algún recurso \textit{online} que ofreciese esta información (para no tener que generarlo manualmente), por lo que se propuso investigar acerca de su disponibilidad.

Otro aspecto que se trató en relación con la API fue la gestión eficiente de las llamadas. Debido a que existe un límite mensual, se propuso la idea de \textbf{almacenar temporalmente} (``\textit{cachear}'') la \textbf{información de los productos} en nuestra base de datos local. De esta forma, si un usuario consultara un producto que ya hubiese sido buscado previamente, el sistema recuperaría los datos de nuestro almacén local en lugar de hacer una nueva solicitud a la API. Para asegurar que la información esté siempre actualizada, los datos almacenados podrían tener un \textbf{periodo de validez}. Una vez superado este tiempo, la aplicación forzaría una nueva llamada a la API para refrescar la información.

Se acordó, para el siguiente \textit{sprint}, comenzar el desarrollo de la aplicación, empezando por la infraestructura de usuarios y roles.

%---------------------------------------------------------------

\subsection{\textit{Sprint} 1}
\label{subsec:A.2_sprint1}

\underline{Duración}: 1 semana (25/09/2025 - 02/10/2025)

Este \textit{sprint} se centró en crear la infraestructura inicial de la aplicación, añadiendo las dependencias y los archivos de configuración necesarios para levantar la aplicación \textit{Spring Boot}. Una vez creada, se empezó a implementar la gestión de usuarios y roles, definiendo las entidades y métodos pertinentes para, como mínimo, realizar operaciones CRUD sobre ellos.

Los \textit{issues} creados para cumplir los objetivos definidos se pueden ver en: \href{https://github.com/aperezolmos/tfg-aperezolmos/milestone/2?closed=1}{\textbf{Sprint 1}}

\subsubsection{Objetivos}

\begin{itemize}
	\item \textbf{Creación del proyecto \textit{Spring Boot}:} Se configuró un proyecto base con las dependencias listadas en <agregar ref>\todo{Agregar ref de apartado dependencias}.
	\item \textbf{Generación del \texttt{docker-compose.yml}:} Se creó dicho archivo para orquestar los contenedores de \textit{MySQL} y \textit{Spring Boot}. La imagen de la aplicación \textit{Spring Boot} se creó a partir de su propio \texttt{Dockerfile}. Se añadió un archivo de inicialización de la base de datos y se comprobó que ambos servicios podían comunicarse entre sí.
	\item \textbf{Creación de las entidades \texttt{User} y \texttt{Role} \todo{Referencia a anexo C -> diseño de datos¿?}:} Se definieron las entidades JPA y las tablas SQL del archivo de inicialización. Se estableció inicialmente que los usuarios solo puedan tener un rol. En iteraciones futuras, se agregará la funcionalidad de tener múltiples roles.
	\item \textbf{Implementación de operaciones CRUD sobre usuarios y roles:} Se agregaron los repositorios, servicios y controladores necesarios para llevar a cabo estas funcionalidades.
	\item \textbf{Configuración de \textit{Spring Security}:} Se adaptaron las entidades \texttt{User} para poder autenticarse en la aplicación mediante \textit{Spring Security} y controlar el acceso a cada \textit{endpoint} según los roles.
	\item \textbf{Creación de formulario de \textit{login} básico:} Un inicio de sesión mediante usuario y contraseña, en el que se muestran mensajes descriptivos en caso de error.
	\item \textbf{Generación de pruebas de integración:} Se generaron pruebas básicas de integración para el repositorio y servicio de usuarios. Sirvieron para familiarizarse con los tipos de pruebas y anotaciones de \textit{Spring Boot}.
\end{itemize}

\imagenConEtiqueta{img/apen_A/A2_sprint1_burndown.png}{Gráfico \textit{burndown} del Sprint 1}{A2_sprint1_burndown}

\subsubsection{Revisión}

Se acordó, para el siguiente \textit{sprint}, seguir implementando los aspectos restantes relativos a la gestión de usuarios en la aplicación, para poder dar prácticamente por acabada esa funcionalidad. Se comentó también el tema de la \textbf{internacionalización} de la aplicación, y cómo sería beneficioso intentar implementarlo desde el primer momento.

%---------------------------------------------------------------

\subsection{\textit{Sprint} 2}
\label{subsec:A.2_sprint2}

\underline{Duración}: 1 semana (02/10/2025 - 09/10/2025)

Este \textit{sprint} se enfocó en expandir la funcionalidad de gestión de usuarios mediante la creación de formularios, como el de registro, edición o mantenimiento de usuarios (por parte de un administrador). También se empezaron a escribir los apéndices A y B de la documentación.

Los \textit{issues} creados para cumplir los objetivos definidos se pueden ver en: \href{https://github.com/aperezolmos/tfg-aperezolmos/milestone/3?closed=1}{\textbf{Sprint 2}}

\subsubsection{Objetivos}

\begin{itemize}
	\item \textbf{Creación de formulario de registro de usuarios:} Una pantalla que permite a nuevos usuarios crear su perfil en la aplicación (con \texttt{ROLE\_USER} por defecto), en la que se muestran mensajes descriptivos en caso de error.
	\item \textbf{Creación de formulario de edición de usuarios:} Una pantalla que permite editar el perfil de un usuario, disponible para el propio usuario y para administradores que quieran editar dicho usuario. Los campos mostrados dependen de la identidad y rol del \textit{usuario editor}.
	\item \textbf{Implementación de módulo de gestión de usuarios para \textit{admins}:} Se creó una página que permite a los administradores de la aplicación ver un listado de todos los usuarios de la misma y poder realizar operaciones de gestión, como la creación, modificación y eliminación de dichos usuarios.
	\item \textbf{Comienzo de la documentación de \textit{sprints} anteriores:} Se redactó acerca de los \textit{sprints} completados hasta entonces (véanse \nameref{subsec:A.2_sprint0} y \nameref{subsec:A.2_sprint1}), para mantener un seguimiento claro de las tareas completadas y las decisiones tomadas durante el ciclo de vida del proyecto.
	\item \textbf{Primera definición de requisitos funcionales:} Aunque estos requisitos ya se tenían presentes desde el comienzo del proyecto, se redactó una primera versión de éstos en la documentación.
\end{itemize}

\imagenConEtiqueta{img/apen_A/A2_sprint2_burndown.png}{Gráfico \textit{burndown} del Sprint 2}{A2_sprint2_burndown}

\subsubsection{Revisión}

Al haber mostrado el funcionamiento de los \textit{endpoints} de la aplicación al tutor mediante \textit{Postman}, se propuso la posible generación de una \textbf{colección de solicitudes} completa, para adjuntar como archivo complementario en cada \textit{release}.

Se acordó, para el siguiente \textit{sprint}, dejar a un lado la funcionalidad de usuarios y proseguir con la gestión de ingredientes y productos ofertados. 

%---------------------------------------------------------------

\subsection{\textit{Sprint} 3}
\label{subsec:A.2_sprint3}

\underline{Duración}: 2 semanas (09/10/2025 - 23/10/2025)

En este \textit{sprint} se comenzó a desarrollar la gestión de los productos ofertados, tomando como referencia las clases existentes en <ref NutriMenu>\todo{Agregar ref}. También se finalizaron aspectos relacionados con la gestión de usuarios, que se consideraban necesarios de implementar lo antes posible. Inicialmente se planificó la duración para 1 semana, pero acabó alargándose debido a cuestiones personales. 

Los \textit{issues} creados para cumplir los objetivos definidos se pueden ver en: \href{https://github.com/aperezolmos/tfg-aperezolmos/milestone/4?closed=1}{\textbf{Sprint 3}}

\subsubsection{Objetivos}

\begin{itemize}
	\item \textbf{Creación de las entidades \texttt{Food}, \texttt{Product} y \texttt{Recipe} \todo{Referencia a anexo C -> diseño de datos¿?}:} Se definieron las entidades JPA, basándose en los campos existentes en las clases de <ref NutriMenu>\todo{Agregar ref} y agregando nuevos. Se simplificaron considerablemente el código y la lógica relacionada con dichas entidades. Las mejoras se explican con más detalle en el \textit{pull request} \href{https://github.com/aperezolmos/tfg-aperezolmos/pull/51}{\#51}.
	\item \textbf{Implementación de operaciones CRUD sobre alimentos y productos:} Se agregaron los repositorios y servicios necesarios para llevar a cabo estas funcionalidades. 
	\item \textbf{Cálculo dinámico de la información nutricional:} Se simplificó enormemente la lógica original, logrando que los cálculos de información nutricional (ya sea para alimentos con cantidades particulares o para productos completos) sean dinámicos y \textbf{no se almacenen en la BBDD}. Las mejoras se explican con más detalle en el \textit{pull request} \href{https://github.com/aperezolmos/tfg-aperezolmos/pull/52}{\#52}.
	\item \textbf{Transformación de la relación \texttt{User-Role} a \texttt{\char`@{}ManyToMany}:} Se modificó la asignación de roles, para que los usuarios pudieran tener más de uno, lo cual permite la creación de roles y funcionalidades más específicas. Se garantiza que, como mínimo, un usuario registrado tendrá \texttt{ROLE\_USER}.
	\item \textbf{Refactorización de los DTOs de \texttt{User}:} Se redujo el número de DTOs para usuarios, fusionando todos en \texttt{UserRequestDTO}\todo{ref a apartado diseño datos?} y manejando las restricciones según formulario mediante \textbf{grupos de validación} (\textit{OnCreate}, \textit{OnUpdate}, etc.).
	\item \textbf{Creación de un \texttt{UserDataInitializer}:} Este componente crea roles y un usuario por defecto en la aplicación.
	\item \textbf{Definición de casos de uso para gestión de usuarios:} Se redactó una primera versión de casos de uso para esta funcionalidad, teniendo en cuenta las acciones que pueden realizar los usuarios según sus roles.
\end{itemize}

\imagenConEtiqueta{img/apen_A/A2_sprint3_burndown.png}{Gráfico \textit{burndown} del Sprint 3}{A2_sprint3_burndown}

\subsubsection{Revisión}

Se acordó plasmar las mejoras respecto al TFG anterior en el apartado de <Aspectos Relevantes?>\todo{Referencia cuando se haya añadido}, para reflejar el proceso de transformación seguido y las decisiones que llevaron a él.

Respecto a la arquitectura de la aplicación, se le planteó al tutor el \textbf{cambio de la arquitectura actual} monolítica, por una \textbf{arquitectura SPA} con \textit{React}. Inicialmente, se había optado por usar \textit{Thymeleaf} como motor de plantillas, y gestionar las vistas de la aplicación mediante controladores web. Sin embargo, la generación de las plantillas HTML y los controladores específicos para cada acción (p. ej. \texttt{UserEditController} o \texttt{UsermanagementController}) resultaban un poco pesados y menos fáciles de escalar. Es por eso que se optó por la naturaleza reutilizable de los componentes de \textit{React} para, en primera instancia, definir de forma más simple las tablas de gestión CRUD para cada entidad (y muchas otras mejoras). \todo{Ref a apartado donde se detalle esto?}

Debido a que la idea del cambio de arquitectura surgió durante el desarrollo del \textit{sprint}, hubo un \textit{issue} que se pospuso para futuras iteraciones, relacionado con la \textbf{internacionalización} de la aplicación. Esto es debido a que la implementación de dicha funcionalidad cambia por completo dependiendo de la arquitectura. Por lo tanto, se pospuso para cuando se disponga de un \textit{frontend} funcional.

Se acordó enfocar el siguiente \textit{sprint} en la \textbf{migración a una arquitectura SPA con \textit{React}}, lo cual conlleva la refactorización de la aplicación \textit{Spring Boot} para que actúe como API REST.

%---------------------------------------------------------------

\subsection{\textit{Sprint} 4}
\label{subsec:A.2_sprint4}

\underline{Duración}: 2 semanas (23/10/2025 - 06/11/2025)

Este \textit{sprint} se dedicó exclusivamente a la migración a \textit{React}. Se refactorizó el proyecto de \textit{Spring Boot} para convertirlo en API REST pura y se eliminaron las dependencias de \textit{Thymeleaf}. Se construyó el nuevo proyecto de \textit{React} y se reimplementaron las funcionalidades existentes hasta la fecha.

Los \textit{issues} creados para cumplir los objetivos definidos se pueden ver en: \href{https://github.com/aperezolmos/tfg-aperezolmos/milestone/5?closed=1}{\textbf{Sprint 4}}

\subsubsection{Objetivos}

\begin{itemize}
	\item \textbf{Investigación de mejores prácticas para la migración de arquitectura:} Abarcó el aprendizaje de las bases de \textit{React} y la selección de librerías útiles para el \textit{frontend}. También se investigó sobre las prácticas estándar para refactorizar la API de \textit{Spring Boot} y sobre los componentes necesarios para la comunicación entre ambos servicios.
	\item \textbf{Refactorización del \textit{backend}:} Se completó la migración completa a un diseño RESTful, eliminando toda la infraestructura web y reescribiendo los servicios y controladores para operar mediante DTOs y peticiones HTTP. Los cambios se explican con más detalle en el \textit{pull request} \href{https://github.com/aperezolmos/tfg-aperezolmos/pull/60}{\#60}.
	\item \textbf{Configuración de la autenticación y CORS:} Se implementaron el servicio y controlador de autenticación (\textit{login} y registro) en la API y se configuró CORS para permitir la comunicación entre la API y el \textit{frontend} de \textit{React}, estableciendo sesiones \textit{stateless}.
	\item \textbf{Creación del proyecto y contenedor \textit{Docker} para \textit{React}:} Para generar el proyecto inicial se utilizó la herramienta de construcción \textit{Vite}. \todo{Ref?} Se creó el \texttt{Dockerfile} para la construcción del contenedor y se agregó el servicio a \texttt{docker-compose.yml}.
	\item \textbf{Implementación de la autenticación en \textit{React}:} Se creó un contexto de autenticación global para manejar el usuario de cada sesión. Se crearon las páginas para \textit{login} y registro de usuarios. 
	\item \textbf{Implementación de la gestión de usuarios en \textit{React}:} Se reimplementaron las funcionalidades existentes acerca del manejo de usuarios y roles, junto con las interfaces asociadas. Las novedades se explican con más detalle en el \textit{pull request} \href{https://github.com/aperezolmos/tfg-aperezolmos/pull/74}{\#74}.
\end{itemize}

\imagenConEtiqueta{img/apen_A/A2_sprint4_burndown.png}{Gráfico \textit{burndown} del Sprint 4}{A2_sprint4_burndown}

\subsubsection{Revisión}

Se acordó, para el siguiente \textit{sprint}, continuar con la creación de páginas de gestión para las entidades restantes (gestión de productos ofertados).

\todo[inline]{Agregar más objetivos del sprint...}

%---------------------------------------------------------------

\subsection{\textit{Sprint} 6}
\label{subsec:A.2_sprint6}

\underline{Duración}: 2 semanas (20/11/2025 - 04/12/2025)

Este \textit{sprint} se utilizó para aplicar unas últimas mejoras a la API REST y desarrollar tests que comprueben el correcto funcionamiento de los servicios que ofrece. También se implementaron las interfaces de gestión restantes de las entidades añadidas en el anterior \textit{sprint}. En general, se preparó la aplicación para su primer \textbf{\textit{release}}. \todo{si se hace un apartado sobre ello, referenciarlo aquí}

Los \textit{issues} creados para cumplir los objetivos definidos se pueden ver en: \href{https://github.com/aperezolmos/tfg-aperezolmos/milestone/7?closed=1}{\textbf{Sprint 6}}

\subsubsection{Objetivos}

\begin{itemize}
	\item \textbf{Refactorización, limpieza y mejora de aspectos puntuales en la API:} Se ``estandarizó'' la API \textit{Spring Boot} para que las entidades y componentes fuesen uniformes, consistentes y que siguieran un patrón similar. Entre las modificaciones se encuentran la incorporación de campos de auditoría (\texttt{createdAt} y \texttt{updatedAt}), la simplificación y eliminación de algunos métodos, la refactorización de los \textit{mappers} para aprovechar mejor la lógica de conversiones, etc.
	\item \textbf{Creación de pruebas unitarias para \textit{mappers}:} Al tener lógica personalizada, se hacía necesario probar los \textit{mappers} generados por \textit{Mapstruct}, para comprobar que la conversión DTO $\leftrightarrow$ Entidad funciona de la manera esperada. Como los comportamientos de los componentes involucrados son fáciles de simular, se utilizó \textit{Mockito} para generar las pruebas, puesto que proporciona tests muy rápidos y aislamiento completo.
	\item \textbf{Creación de pruebas unitarias para servicios:} Se probó la lógica de negocio (validaciones, excepciones lanzadas, métodos de consulta) de todas las entidades existentes en la API. Se utilizó \textit{Mockito} por su rapidez y aislamiento.
	\item \textbf{Creación de pruebas de integración para servicios:} Se crearon pruebas con dependencias reales utilizando \texttt{\char`@{}SpringBootTest} y transaccionalidad, para probar la \textbf{persistencia}, relaciones entre entidades y conflictos que solo se dan en casos ``reales''. Estas pruebas de integración son más lentas puesto que necesitan cargar todo el contexto \textit{Spring}. Al ser más costosas, solo se crearon en entidades más complejas o que ``gestionan a otras'' (normalmente, las \textbf{entidades propietarias} de la relación).
	\item \textbf{Implementación de la gestión de productos en \textit{React}:} Se crearon páginas y componentes para visualizar listados, crear, editar y eliminar alimentos y productos, junto con su información nutricional. Se reutilizaron los componentes y \textit{layouts} de gestión creados anteriormente. Las novedades se explican con más detalle en el \textit{pull request} \href{https://github.com/aperezolmos/tfg-aperezolmos/pull/107}{\#107}.
	\item \textbf{Implementación de la gestión de reservas en \textit{React}:} De igual forma, se crearon páginas y componentes para gestión básica con operaciones CRUD sobre mesas y reservas. Las novedades se explican con más detalle en el \textit{pull request} \href{https://github.com/aperezolmos/tfg-aperezolmos/pull/106}{\#106}.
	\item \textbf{Creación de \textit{dashboard} para gestionar pedidos:} Se creó una vista general que muestra tanto un listado de pedidos pendientes como un listado de productos ofertados. Se permite añadir productos al pedido actual haciendo \textit{click} en ellos y se permite alternar entre pedidos de bar o comedor.
\end{itemize}

%\imagenConEtiqueta{img/apen_A/A2_sprint6_burndown.png}{Gráfico \textit{burndown} del Sprint 6}{A2_sprint6_burndown}

\subsubsection{Revisión}

\todo[inline]{Redactar...}

%%%%%%%%%%%%%%%%%%%%%%%%%%%%%%%%%%%%%%%%%%%%%%%%%%%%%%%%%%%%%%%%

\section{Estudio de viabilidad}

%%%%%%%%%%%%%%%%%%%%%%%%%%%%%%%%%%%%%%%%%%%%%%%%%%%%%%%%%%%%%%%%

\subsection{Viabilidad económica}

%%%%%%%%%%%%%%%%%%%%%%%%%%%%%%%%%%%%%%%%%%%%%%%%%%%%%%%%%%%%%%%%

\subsection{Viabilidad legal}


