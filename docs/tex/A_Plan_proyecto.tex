\apendice{Plan de Proyecto Software}

\section{Introducción}

\todo[inline]{Pendiente escribir.}

%%%%%%%%%%%%%%%%%%%%%%%%%%%%%%%%%%%%%%%%%%%%%%%%%%%%%%%%%%%%%%%%

\section{Planificación temporal}

\todo[inline]{Pendiente escribir.}

%%%%%%%%%%%%%%%%%%%%%%%%%%%%%%%%%%%%%%%%%%%%%%%%%%%%%%%%%%%%%%%%

\subsection{\textit{Sprint} 0}
\label{subsec:A.2_sprint0}

\underline{Duración}: 1 semana (18/09/2025 - 25/09/2025)

Este primer \textit{sprint} abarca las tareas de investigación y definición inicial, enfocadas en determinar el alcance mínimo del proyecto, junto con las tecnologías y recursos a utilizar.

Los \textit{issues} creados para cumplir los objetivos definidos se pueden ver en: \href{https://github.com/aperezolmos/tfg-aperezolmos/milestone/1?closed=1}{\textbf{Sprint 0}}

\subsubsection{Objetivos}

\begin{itemize}
	\item \textbf{Creación del repositorio:} Se configuró el repositorio de \textit{GitHub} para el control de versiones. Utilizando \textit{GitHub Projects}, se asoció un proyecto a dicho repositorio para la gestión y mantenimiento de las tareas de cada \textit{sprint}.
	\item \textbf{Creación de los \textit{epics}\todo{ref de qué son¿?}:} Se definieron \textit{issues} principales para estructurar el proyecto en áreas funcionales (desarrollo, \textit{testing}, documentación) y agrupar tareas relacionadas, agilizando el acceso y el seguimiento del progreso.
	\item \textbf{Definición del \textit{stack} tecnológico:} Se eligió \textit{\textbf{Spring Boot}} (con \textit{Spring Security} y \textit{Spring Data JPA}) para el \textit{backend} y la lógica de negocio, junto con \textit{\textbf{MySQL}} como base de datos. Para el \textit{frontend}, se optó por \textit{\textbf{Thymeleaf}}, que facilita la generación de vistas HTML dinámicas desde el servidor. Finalmente, se seleccionó \textit{\textbf{Docker}} para los contenedores de la aplicación y la base de datos, asegurando un despliegue portable y reproducible.
	\item \textbf{Estudio y evaluación de APIs nutricionales:} Se realizó una comparativa entre varias APIs de alimentos. La API elegida fue \textit{Nutritionix}\todo{ref?}. Los criterios de elección y el resto de candidatas pueden verse en el \textit{issue} \href{https://github.com/aperezolmos/tfg-aperezolmos/issues/10}{\#10}.
	\item \textbf{Revisión de trabajos previos y TFGs relacionados:} ...NutriMenu...\todo{referencia bibliografía}
	\todo[inline]{Pendiente completar.}
	\item \textbf{Análisis de integración con plataformas de \textit{delivery}:} Se investigó acerca de los requisitos necesarios para obtener claves de acceso a las APIs de distintas plataformas de \textit{delivery}. Se terminó por concluir que esta integración se pospondría para etapas más avanzadas del desarrollo, en caso de que se dispusiera del tiempo necesario.
\end{itemize}

\imagenConEtiqueta{img/apen_A/A2_sprint0_burndown.png}{Gráfico \textit{burndown} del Sprint 0}{A2_sprint0_burndown}

\subsubsection{Revisión}

Como la API de \textit{Nutritionix} no proporciona directamente datos de alérgenos, se valoró la idea de \textbf{``\textit{mapear}'' los alérgenos} en base a los ingredientes, mediante un modelo de reglas y una lista predefinida de alérgenos. Se planteó la posibilidad de que existiese algún recurso \textit{online} que ofreciese esta información (para no tener que generarlo manualmente), por lo que se propuso investigar acerca de su disponibilidad.

Otro aspecto que se trató en relación con la API fue la gestión eficiente de las llamadas. Debido a que existe un límite mensual, se propuso la idea de \textbf{almacenar temporalmente} (``\textit{cachear}'') la \textbf{información de los productos} en nuestra base de datos local. De esta forma, si un usuario consultara un producto que ya hubiese sido buscado previamente, el sistema recuperaría los datos de nuestro almacén local en lugar de hacer una nueva solicitud a la API. Para asegurar que la información esté siempre actualizada, los datos almacenados podrían tener un \textbf{periodo de validez}. Una vez superado este tiempo, la aplicación forzaría una nueva llamada a la API para refrescar la información.

Se acordó, para el siguiente \textit{sprint}, comenzar el desarrollo de la aplicación, empezando por la infraestructura de usuarios y roles.

%---------------------------------------------------------------

\subsection{\textit{Sprint} 1}
\label{subsec:A.2_sprint1}

\underline{Duración}: 1 semana (25/09/2025 - 02/10/2025)

Este \textit{sprint} se centró en crear la infraestructura inicial de la aplicación, añadiendo las dependencias y los archivos de configuración necesarios para levantar la aplicación \textit{Spring Boot}. Una vez creada, se empezó a implementar la gestión de usuarios y roles, definiendo las entidades y métodos pertinentes para, como mínimo, realizar operaciones CRUD sobre ellos.

Los \textit{issues} creados para cumplir los objetivos definidos se pueden ver en: \href{https://github.com/aperezolmos/tfg-aperezolmos/milestone/2?closed=1}{\textbf{Sprint 1}}

\subsubsection{Objetivos}

\begin{itemize}
	\item \textbf{Creación del proyecto \textit{Spring Boot}:} Se configuró un proyecto base con las dependencias listadas en <agregar ref>\todo{Agregar ref de apartado dependencias}.
	\item \textbf{Generación del \texttt{docker-compose.yml}:} Se creó dicho archivo para orquestar los contenedores de \textit{MySQL} y \textit{Spring Boot}. La imagen de la aplicación \textit{Spring Boot} se creó a partir de su propio \texttt{Dockerfile}. Se añadió un archivo de inicialización de la base de datos y se comprobó que ambos servicios podían comunicarse entre sí.
	\item \textbf{Creación de las entidades \texttt{User} y \texttt{Role} \todo{Referencia a anexo C -> diseño de datos¿?}:} Se definieron las entidades JPA y las tablas SQL del archivo de inicialización. Se estableció inicialmente que los usuarios solo puedan tener un rol. En iteraciones futuras, se agregará la funcionalidad de tener múltiples roles.
	\item \textbf{Implementación de operaciones CRUD sobre usuarios y roles:} Se agregaron los repositorios, servicios y controladores necesarios para llevar a cabo estas funcionalidades.
	\item \textbf{Configuración de \textit{Spring Security}:} Se adaptaron las entidades \texttt{User} para poder autenticarse en la aplicación mediante \textit{Spring Security} y controlar el acceso a cada \textit{endpoint} según los roles.
	\item \textbf{Creación de formulario de \textit{login} básico:} Un inicio de sesión mediante usuario y contraseña, en el que se muestran mensajes descriptivos en caso de error.
	\item \textbf{Generación de pruebas de integración:} Se generaron pruebas básicas de integración para el repositorio y servicio de usuarios. Sirvieron para familiarizarse con los tipos de pruebas y anotaciones de \textit{Spring Boot}.
\end{itemize}

\imagenConEtiqueta{img/apen_A/A2_sprint1_burndown.png}{Gráfico \textit{burndown} del Sprint 1}{A2_sprint1_burndown}

\subsubsection{Revisión}

Se acordó, para el siguiente \textit{sprint}, seguir implementando los aspectos restantes relativos a la gestión de usuarios en la aplicación, para poder dar prácticamente por acabada esa funcionalidad. Se comentó también el tema de la \textbf{internacionalización} de la aplicación, y cómo sería beneficioso intentar implementarlo desde el primer momento.

%---------------------------------------------------------------

\subsection{\textit{Sprint} 2}
\label{subsec:A.2_sprint2}

\underline{Duración}: 1 semana (02/10/2025 - 09/10/2025)

Este \textit{sprint} se enfocó en expandir la funcionalidad de gestión de usuarios mediante la creación de formularios, como el de registro, edición o mantenimiento de usuarios (por parte de un administrador). También se empezaron a escribir los apéndices A y B de la documentación.

Los \textit{issues} creados para cumplir los objetivos definidos se pueden ver en: \href{https://github.com/aperezolmos/tfg-aperezolmos/milestone/3?closed=1}{\textbf{Sprint 2}}

\subsubsection{Objetivos}

\begin{itemize}
	\item \textbf{Creación de formulario de registro de usuarios:} Una pantalla que permite a nuevos usuarios crear su perfil en la aplicación (con \texttt{ROLE\_USER} por defecto), en la que se muestran mensajes descriptivos en caso de error.
	\item \textbf{Creación de formulario de edición de usuarios:} Una pantalla que permite editar el perfil de un usuario, disponible para el propio usuario y para administradores que quieran editar dicho usuario. Los campos mostrados dependen de la identidad y rol del \textit{usuario editor}.
	\item \textbf{Implementación de módulo de gestión de usuarios para \textit{admins}:} Se creó una página que permite a los administradores de la aplicación ver un listado de todos los usuarios de la misma y poder realizar operaciones de gestión, como la creación, modificación y eliminación de dichos usuarios.
	\item \textbf{Comienzo de la documentación de \textit{sprints} anteriores:} Se redactó acerca de los \textit{sprints} completados hasta entonces (véanse \nameref{subsec:A.2_sprint0} y \nameref{subsec:A.2_sprint1}), para mantener un seguimiento claro de las tareas completadas y las decisiones tomadas durante el ciclo de vida del proyecto.
	\item \textbf{Primera definición de requisitos funcionales:} Aunque estos requisitos ya se tenían presentes desde el comienzo del proyecto, se redactó una primera versión de éstos en la documentación.
\end{itemize}

\imagenConEtiqueta{img/apen_A/A2_sprint2_burndown.png}{Gráfico \textit{burndown} del Sprint 2}{A2_sprint2_burndown}

\subsubsection{Revisión}

Al haber mostrado el funcionamiento de los \textit{endpoints} de la aplicación al tutor mediante \textit{Postman}, se propuso la posible generación de una \textbf{colección de solicitudes} completa, para adjuntar como archivo complementario en cada \textit{release}.

Se acordó, para el siguiente \textit{sprint}, dejar a un lado la funcionalidad de usuarios y proseguir con la gestión de ingredientes y productos ofertados. 

%%%%%%%%%%%%%%%%%%%%%%%%%%%%%%%%%%%%%%%%%%%%%%%%%%%%%%%%%%%%%%%%

\section{Estudio de viabilidad}

%%%%%%%%%%%%%%%%%%%%%%%%%%%%%%%%%%%%%%%%%%%%%%%%%%%%%%%%%%%%%%%%

\subsection{Viabilidad económica}

%%%%%%%%%%%%%%%%%%%%%%%%%%%%%%%%%%%%%%%%%%%%%%%%%%%%%%%%%%%%%%%%

\subsection{Viabilidad legal}


