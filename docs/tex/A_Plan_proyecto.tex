\apendice{Plan de Proyecto Software}

\section{Introducción}

\todo[inline]{Pendiente escribir.}

%%%%%%%%%%%%%%%%%%%%%%%%%%%%%%%%%%%%%%%%%%%%%%%%%%%%%%%%%%%%%%%%

\section{Planificación temporal}

\todo[inline]{Pendiente escribir.}

%%%%%%%%%%%%%%%%%%%%%%%%%%%%%%%%%%%%%%%%%%%%%%%%%%%%%%%%%%%%%%%%

\subsection{\textit{Sprint} 1}

Duración: 25/09/2025 - 02/10/2025

Este \textit{sprint} se centró en crear la infraestructura inicial de la aplicación, añadiendo las dependencias y los archivos de configuración necesarios para levantar la aplicación \textit{Spring Boot}. Una vez creada, se empezó a implementar la gestión de usuarios y roles, definiendo las entidades y métodos pertinentes para, como mínimo, realizar operaciones CRUD sobre ellos.

Los \textit{issues} creados para cumplir los objetivos definidos se pueden ver en: \href{https://github.com/aperezolmos/tfg-aperezolmos/milestone/2?closed=1}{\textbf{Sprint 1}}

\subsubsection{Objetivos}

\begin{itemize}
	\item \textbf{Creación del proyecto \textit{Spring Boot}:} Se configuró un proyecto base con las dependencias listadas en <agregar ref>\todo{Agregar ref de apartado dependencias}.
	\item \textbf{Generación del \texttt{docker-compose.yml}:} Se creó dicho archivo para orquestar los contenedores de \textit{MySQL} y \textit{Spring Boot}. La imagen de la aplicación \textit{Spring Boot} se creó a partir de su propio \texttt{Dockerfile}. Se añadió un archivo de inicialización de la base de datos y se comprobó que ambos servicios podían comunicarse entre sí.
	\item \textbf{Creación de las entidades \texttt{User} y \texttt{Role} \todo{Referencia a anexo C -> diseño de datos¿?}:} Se definieron las entidades JPA y las tablas SQL del archivo de inicialización. Se estableció inicialmente que los usuarios solo puedan tener un rol. En iteraciones futuras, se agregará la funcionalidad de tener múltiples roles.
	\item \textbf{Implementación de operaciones CRUD sobre usuarios y roles:} Se agregaron los repositorios, servicios y controladores necesarios para llevar a cabo estas funcionalidades.
	\item \textbf{Configuración de \textit{Spring Security}:} Se adaptaron las entidades \texttt{User} para poder autenticarse en la aplicación mediante \textit{Spring Security} y controlar el acceso a cada \textit{endpoint} según los roles.
	\item \textbf{Creación de formulario de \textit{login} básico:} Un inicio de sesión mediante usuario y contraseña, en el que se muestran mensajes descriptivos en caso de error.
	\item \textbf{Generación de pruebas de integración:} Se generaron pruebas básicas de integración para el repositorio y servicio de usuarios. Sirvieron para familiarizarse con los tipos de pruebas y anotaciones de \textit{Spring Boot}.
\end{itemize}

\missingfigure{Lista de issues del sprint 1? O un gráfico de commits o burn down}

\subsubsection{Revisión}

Se acordó, para el siguiente \textit{sprint}, seguir implementando los aspectos restantes relativos a la gestión de usuarios en la aplicación, para poder dar prácticamente por acabada esa funcionalidad. Se comentó también el tema de la \textbf{internacionalización} de la aplicación, y cómo sería beneficioso intentar implementarlo desde el primer momento.

%%%%%%%%%%%%%%%%%%%%%%%%%%%%%%%%%%%%%%%%%%%%%%%%%%%%%%%%%%%%%%%%

\section{Estudio de viabilidad}

%%%%%%%%%%%%%%%%%%%%%%%%%%%%%%%%%%%%%%%%%%%%%%%%%%%%%%%%%%%%%%%%

\subsection{Viabilidad económica}

%%%%%%%%%%%%%%%%%%%%%%%%%%%%%%%%%%%%%%%%%%%%%%%%%%%%%%%%%%%%%%%%

\subsection{Viabilidad legal}


