\apendice{Anexo de sostenibilización curricular}

\section{Introducción}

Este apéndice tiene como objetivo reflexionar sobre los aspectos de sostenibilidad abordados en el Trabajo de Fin de Grado y analizar cómo el desarrollo de la aplicación web \textit{Orderly} se alinea con los Objetivos de Desarrollo Sostenible (ODS) definidos por la Agenda 2030 de las Naciones Unidas~\cite{un:ODS}. Esta reflexión se realiza desde una perspectiva tanto técnica como social, atendiendo a las competencias de sostenibilidad adquiridas durante la formación académica y aplicadas de manera práctica en el proyecto. A continuación, se detallan los ODS que más se alinean con el propósito de \textit{Orderly}.

\section{ODS 3: Salud y bienestar}

La aplicación permite consultar \textbf{información nutricional} detallada de los alimentos, facilitando el acceso a datos que ayudan a tomar decisiones alimentarias más informadas. La visualización de indicadores como el \textit{\textbf{Nutri-Score}} y el \textbf{grupo NOVA}, junto con tablas y gráficos de nutrientes, contribuye a promover hábitos de consumo más saludables. 

Adicionalmente, la posibilidad de \textbf{filtrar alérgenos} resulta especialmente relevante para personas con intolerancias o alergias alimentarias, mejorando la seguridad alimentaria y reduciendo riesgos para la salud.

\section{ODS 8: Trabajo decente y crecimiento económico}

\textit{Orderly} busca \textbf{optimizar y simplificar} la gestión diaria de un local de restauración mediante la digitalización de procesos que, tradicionalmente, se realizan de forma manual o dispersa. La \textbf{centralización} de la información reduce la carga administrativa del personal, mejora la organización del trabajo y disminuye errores operativos. Esto puede traducirse en un entorno laboral más eficiente y menos estresante, favoreciendo \textbf{condiciones de trabajo más dignas y productivas}, especialmente en un sector caracterizado por ritmos elevados y alta rotación de personal.

Por otro lado, \textit{Orderly} representa una oportunidad de \textbf{transformación digital} para pequeños negocios. Ayudar a que un \textbf{restaurante local} pueda dar el salto de las notas en papel a una gestión web robusta contribuye a su competitividad y, por tanto, a la estabilidad económica del negocio a largo plazo.

\section{ODS 9: Industria, innovación e infraestructura} 

El desarrollo de la aplicación se apoya en tecnologías modernas y ampliamente utilizadas en la industria del \textit{software}, como \textit{React}, \textit{Spring Boot}, \textit{APIs REST} y contenedores \textit{Docker}. La arquitectura modular y el uso de estándares abiertos favorecen la \textbf{escalabilidad}, el mantenimiento y la \textbf{reutilización} del sistema. Además, la \textbf{integración de servicios externos} de datos abiertos, como \textit{Open Food Facts}, refleja un enfoque innovador que promueve el aprovechamiento de infraestructuras digitales existentes y el \textbf{acceso abierto} a la información.

\section{ODS 12: Producción y consumo responsables}

Aunque \textit{Orderly} actualmente no gestiona el \textit{stock} de productos, su labor en la digitalización de la operativa diaria tiene un impacto directo en el consumo de recursos.

El beneficio más evidente es la \textbf{transición} hacia un entorno \textbf{``sin papel''}. En muchos restaurantes, el ciclo de vida de un \textit{ticket} de comanda es de apenas unos minutos antes de acabar en la papelera. Multiplicando esto por cientos de servicios al año, el \textbf{ahorro de papel} que supone gestionar comandas y reservas de forma digital es significativo.

Además, al proporcionar información sobre el \textbf{grupo NOVA} (grado de procesamiento) y el perfil nutricional de los alimentos, la aplicación fomenta de forma indirecta un consumo más responsable. Un restaurante que tiene visibilidad sobre la calidad nutricional de sus materias primas está mejor posicionado para tomar \textbf{decisiones de compra más éticas y saludables}, reduciendo la dependencia de productos industriales que suelen generar más \textbf{residuos de envasado} y procesos productivos menos sostenibles.

\section{ODS 13: Acción por el clima}

Desde una perspectiva técnica y medioambiental, el uso de tecnologías de virtualización mediante contenedores \textit{Docker} contribuye a una \textbf{gestión más eficiente} de los \textbf{recursos computacionales}, alineándose de forma transversal con este objetivo.

La contenerización permite un despliegue más ligero y controlado de la aplicación, optimizando el uso de infraestructuras y \textbf{reduciendo el consumo innecesario} de recursos en comparación con soluciones menos eficientes. Aunque el impacto ambiental directo es limitado en el contexto de un TFG, el enfoque adoptado refleja una conciencia sobre la importancia de la eficiencia tecnológica.

%\section{ODS 4: Educación de calidad}
%
%El propio desarrollo del proyecto ha supuesto la adquisición y aplicación de \textbf{competencias} relacionadas con la sostenibilidad, la ética profesional y el uso responsable de la tecnología. El trabajo ha permitido integrar conocimientos técnicos con una visión social más amplia, comprendiendo el papel del ingeniero informático en la creación de soluciones digitales que aporten valor más allá de lo puramente funcional.

\section{Conclusiones}

Aunque \textit{Orderly} es principalmente una aplicación de gestión para el sector de la restauración, su diseño y funcionalidades permiten alinearla con diversos Objetivos de Desarrollo Sostenible, especialmente aquellos relacionados con la salud, el bienestar, el trabajo digno, la innovación tecnológica y el consumo responsable. 

El desarrollo de este trabajo demuestra que es posible integrar criterios de sostenibilidad en proyectos tecnológicos de alcance moderado, reforzando la idea de que el desarrollo de \textit{software} puede y debe contribuir de forma positiva a la sociedad.
