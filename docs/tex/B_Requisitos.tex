\apendice{Especificación de Requisitos}

\section{Introducción}

\todo[inline]{Pendiente escribir.}

%%%%%%%%%%%%%%%%%%%%%%%%%%%%%%%%%%%%%%%%%%%%%%%%%%%%%%%%%%%%%%%%

\section{Objetivos generales}

\todo[inline]{Pendiente escribir.}

%%%%%%%%%%%%%%%%%%%%%%%%%%%%%%%%%%%%%%%%%%%%%%%%%%%%%%%%%%%%%%%%

\section{Catálogo de requisitos}

\subsection{Requisitos funcionales}

\todo[inline]{Pendiente añadir descripción breve a cada uno.}

\begin{itemize}
	\item \textbf{RF-1 Gestión de productos:} ... 
	\begin{itemize}
		\item \textbf{RF-1.1 Agregar o borrar un producto:} ...
		\item \textbf{RF-1.2 Modificar información de un producto:} ...
		\item \textbf{RF-1.3 Visualizar información nutricional:} ...
	\end{itemize}
	
	\item \textbf{RF-2 Gestión de pedidos:} tanto en barra como en comedor...
	\begin{itemize}
		\item \textbf{RF-2.1 Creación de nuevos pedidos:} ...
		\begin{itemize}
			\item \textbf{RF-2.1.1 Añadir nuevos pedidos:} ...
			\item \textbf{RF-2.1.2 Incrementar/decrementar cantidades pedidas:} ...
			\item \textbf{RF-2.1.3 Actualización automática del precio total del pedido:} ...
		\end{itemize}
		\item \textbf{RF-2.2 Visualización de los pedidos existentes:} ...
		\begin{itemize}
			\item \textbf{RF-2.2.1 Modificación de los pedidos:} ...
			\item \textbf{RF-2.2.2 Cancelación de los pedidos:} ...
		\end{itemize}
		\item \textbf{RF-2.3 Modificación de las raciones ofrecidas:} ...
		\begin{itemize}
			\item \textbf{RF-2.3.1 Añadir/borrar productos:} ...
			\item \textbf{RF-2.3.2 Modificar precio unitario de producto:} ...
		\end{itemize}
	\end{itemize}
	
	\item \textbf{RF-3 Gestión de reservas de mesas:} ...
	\begin{itemize}
		\item \textbf{RF-3.1 Visualización de mesas sin reservar:} ...
		\begin{itemize}
			\item \textbf{RF-3.1.1 Asignación de nuevos comensales a mesas:} ...
		\end{itemize}
		\item \textbf{RF-3.2 Visualización de mesas reservadas:} ...
		\begin{itemize}
			\item \textbf{RF-3.2.1 Visualización del número de mesa y hora de reserva:} ...
			\item \textbf{RF-3.2.2 Modificación de la reserva de la mesa:} ...
			\item \textbf{RF-3.2.3 Cancelación de la reserva de la mesa:} ...
		\end{itemize}
		\item \textbf{RF-3.3 Modificación de las mesas por administrador:} tamaño, cantidad...
	\end{itemize}
	
	\item \textbf{RF-4 Gestión de usuarios:} ...
	\begin{itemize}
		\item \textbf{RF-4.1 Identificación del usuario:} ...
		\begin{itemize}
			\item \textbf{RF-4.1.1 Búsqueda del usuario en el sistema:} ...
			\item \textbf{RF-4.1.2 Validación de contraseña:} ...
		\end{itemize}
		\item \textbf{RF-4.2 Información de los usuarios :} ...
		\begin{itemize}
			\item \textbf{RF-4.2.1 Añadir o borrar un usuario:} ...
			\item \textbf{RF-4.2.2 Modificación parámetros de un usuario:} ...
		\end{itemize}
	\end{itemize}
\end{itemize}

%%%%%%%%%%%%%%%%%%%%%%%%%%%%%%%%%%%%%%%%%%%%%%%%%%%%%%%%%%%%%%%%

\section{Especificación de requisitos}

\todo[inline]{Líneas introductorias.}

\missingfigure{Diagrama de casos de uso}

%---------------------------------------------------------------

\subsection{Actores}

\todo[inline]{Explicar brevemente los actores.}

%---------------------------------------------------------------

\subsection{Casos de uso}

% Caso de Uso 1 -> Registrarse (register).
\BeginCasoUso{tab:CU-1}{CU-1}{Registrarse (\textit{register})}
	\textbf{Versión}              & 1.0    \\
	\textbf{Autor}                & Amanda Pérez Olmos \\
	\textbf{Requisitos asociados} & \todo[inline]{RF-xx, RF-xx} \\
	\textbf{Descripción}          & Visitante crea una nueva cuenta de usuario. \\
	\textbf{Precondición}         & El usuario no está autenticado. \\
	\textbf{Acciones}             &
	\begin{enumerate}
		\def\labelenumi{\arabic{enumi}.}
		\tightlist
		\item Hacer \textit{click} en el botón \textit{Registrarse}.
		\item Rellenar el formulario de registro (como mínimo, el nombre de usuario y la contraseña).
		\item El sistema valida los campos.
		\begin{enumerate}
			\item Si la validación es exitosa, el sistema crea el usuario y le añade el rol por defecto. Se redirige al usuario a la página de inicio de sesión.
			\item Si la validación falla, se muestra un mensaje con la descripción del error.
		\end{enumerate}
	\end{enumerate} \\
	\textbf{Postcondición}        & Nuevo usuario persistido en la BBDD con el rol por defecto. \\
	\textbf{Excepciones}          &
	\begin{itemize}
		\tightlist
		\item El \textit{username} ya existe.
		\item La confirmación de contraseña no coincide con la contraseña introducida.
		\item Alguno de los campos supera la longitud máxima.
		\item No se ha rellenado el campo de \textit{username} o el de contraseña.
	\end{itemize} \\
	\textbf{Importancia}          & Alta \\
\EndCasoUso


% Caso de Uso 2 -> Iniciar sesión (login).
\BeginCasoUso{tab:CU-2}{CU-2}{Iniciar sesión (\textit{login})}
	\textbf{Versión}              & 1.0    \\
	\textbf{Autor}                & Amanda Pérez Olmos \\
	\textbf{Requisitos asociados} & \todo[inline]{RF-xx, RF-xx} \\
	\textbf{Descripción}          & Usuario se autentica y accede a la aplicación. \\
	\textbf{Precondición}         & 
	\begin{itemize}
		\tightlist
		\item El usuario tiene una cuenta existente.
		\item El usuario no está autenticado.
	\end{itemize} \\
	\textbf{Acciones}             &
	\begin{enumerate}
		\def\labelenumi{\arabic{enumi}.}
		\tightlist
		\item Hacer \textit{click} en el botón \textit{Iniciar sesión}.
		\item Introducir \textit{username} y contraseña.
		\item El sistema valida las credenciales.
		\begin{enumerate}
			\item Si la validación es exitosa, el sistema crea el usuario y le añade el rol por defecto.
			\item Si la validación falla, se muestra un mensaje con la descripción del error.
		\end{enumerate}
	\end{enumerate} \\
	\textbf{Postcondición}        & 
	\begin{itemize}
		\tightlist
		\item Usuario autenticado y sesión activa.
		\item Acceso a vistas según roles.
	\end{itemize} \\
	\textbf{Excepciones}          & Credenciales incorrectas o vacías. \\
	\textbf{Importancia}          & Alta \\
\EndCasoUso


% Caso de Uso 3 -> Cerrar sesión (logout).
\BeginCasoUso{tab:CU-3}{CU-3}{Cerrar sesión (\textit{logout})}
	\textbf{Versión}              & 1.0    \\
	\textbf{Autor}                & Amanda Pérez Olmos \\
	\textbf{Requisitos asociados} & \todo[inline]{RF-xx, RF-xx} \\
	\textbf{Descripción}          & Usuario termina su sesión y sale de la aplicación. \\
	\textbf{Precondición}         & El usuario está autenticado. \\
	\textbf{Acciones}             &
	\begin{enumerate}
		\def\labelenumi{\arabic{enumi}.}
		\tightlist
		\item Hacer \textit{click} en el botón \textit{Cerrar sesión} de la barra de navegación (\textit{navbar}).
		\item El sistema invalida la sesión y redirige a la página de inicio.
		\todo[inline]{inicio o login?}
	\end{enumerate} \\
	\textbf{Postcondición}        & Sesión inválida. \\
	\textbf{Excepciones}          & - \\
	\textbf{Importancia}          & Baja \\
\EndCasoUso


% Caso de Uso 4 -> Editar perfil propio.
\BeginCasoUso{tab:CU-4}{CU-4}{Editar perfil propio}
	\textbf{Versión}              & 1.0    \\
	\textbf{Autor}                & Amanda Pérez Olmos \\
	\textbf{Requisitos asociados} & \todo[inline]{RF-xx, RF-xx} \\
	\textbf{Descripción}          & Usuario edita la información de su perfil. \\
	\textbf{Precondición}         & El usuario está autenticado. \\
	\textbf{Acciones}             &
	\begin{enumerate}
		\def\labelenumi{\arabic{enumi}.}
		\tightlist
		\item Hacer \textit{click} en el botón \textit{Editar perfil} de la barra de navegación (\textit{navbar}).
		\item Modificar campos permitidos.
		\begin{enumerate}
			\item Solo podrá editar sus roles si tiene permisos de administrador.
			\item Si se quiere editar la contraseña, se deberá introducir la contraseña actual.
		\end{enumerate}
		\item El sistema valida y persiste los cambios.
		\begin{enumerate}
			\item Si se ha editado el \textit{username}, se cierra sesión automáticamente.
		\end{enumerate}
	\end{enumerate} \\
	\textbf{Postcondición}        & Campos del usuario actualizados. \\
	\textbf{Excepciones}          & 
	\begin{itemize}
		\tightlist
		\item El \textit{username} ya existe.
		\item La confirmación de contraseña no coincide con la contraseña introducida.
		\item La contraseña introducida como ``\textit{contraseña actual}'' no coincide con la real.
		\item Alguno de los campos supera la longitud máxima.
	\end{itemize} \\
	\textbf{Importancia}          & Media \\
\EndCasoUso


% Caso de Uso ? -> Recuperar contraseña.


% Caso de Uso 5 -> Gestionar usuarios (admin).
\todo{Caso general o casos separados?}
\BeginCasoUso{tab:CU-5}{CU-5}{Gestionar usuarios (admin)}
	\textbf{Versión}              & 1.0    \\
	\textbf{Autor}                & Amanda Pérez Olmos \\
	\textbf{Requisitos asociados} & \todo[inline]{RF-xx, RF-xx} \\
	\textbf{Descripción}          & Administrador accede a vista que muestra todos los usuarios y puede crear, editar o eliminar cualquiera de ellos. \\
	\textbf{Precondición}         & El usuario está autenticado y tiene rol de administrador. \\
	\textbf{Acciones}             &
	\begin{enumerate}
		\def\labelenumi{\arabic{enumi}.}
		\tightlist
		\item \textbf{Listar usuarios}: Solicitar listado con datos relevantes (sin campos sensibles) de cada usuario.
		\item \textbf{Crear usuarios}: Rellenar formulario con datos básicos, incluyendo la asignación de roles.
		\item \textbf{Editar usuarios}: Modificar cualquier campo editable del formulario, incluyendo la adición y eliminación de roles.
		\item \textbf{Eliminar usuarios}: Solicitar la eliminación de un usuario, con confirmación.
	\end{enumerate} \\
	\textbf{Postcondición}        & 
	\begin{itemize}
		\tightlist
		\item Operaciones reflejadas en la BBDD.
		\item Cambios de roles afectan permisos en nuevas sesiones.
	\end{itemize} \\
	\textbf{Excepciones}          & 
	\begin{itemize}
		\tightlist
		\item Acceso no autorizado.
		\item \textit{Username} duplicado (en creación o edición).
		\item Intento de eliminar o desprivilegiar al último administrador activo.
	\end{itemize} \\
	\textbf{Importancia}          & Alta \\
\EndCasoUso


% Caso de Uso 6 -> Gestionar roles (admin).
% + crear, editar y eliminar + asignar/desasignar a usuarios.
\BeginCasoUso{tab:CU-6}{CU-6}{Gestionar roles (admin)}
	\textbf{Versión}              & 1.0    \\
	\textbf{Autor}                & Amanda Pérez Olmos \\
	\textbf{Requisitos asociados} & \todo[inline]{RF-xx, RF-xx} \\
	\textbf{Descripción}          & Administrador accede a vista que muestra todos los roles y puede crear, editar o eliminar cualquiera de ellos. \\
	\textbf{Precondición}         & El usuario está autenticado y tiene rol de administrador. \\
	\textbf{Acciones}             &
	\begin{enumerate}
		\def\labelenumi{\arabic{enumi}.}
		\tightlist
		\item \textbf{Crear roles}: Agregar un nuevo rol, otorgándole un nombre y descripción.
		\item \textbf{Editar roles}: Editar el nombre o descripción de roles existentes.
		\item \textbf{Eliminar roles}: Borrar roles existentes que no estén asignados a ningún usuario.
	\end{enumerate} \\
	\textbf{Postcondición}        & 
	\begin{itemize}
		\tightlist
		\item Operaciones reflejadas en la BBDD.
		\item Cambios de roles afectan permisos en nuevas sesiones.
	\end{itemize} \\
	\textbf{Excepciones}          & 
	\begin{itemize}
		\tightlist
		\item Acceso no autorizado.
		\item Nombre duplicado (en creación o edición).
		\item Rol está asignado a un usuario y no puede eliminarse.
	\end{itemize} \\
	\textbf{Importancia}          & Media \\
\EndCasoUso
