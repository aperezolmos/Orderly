\apendice{Especificación de Requisitos}

\section{Introducción}

Una muestra de cómo podría ser una tabla de casos de uso:

% Caso de Uso 1 -> Consultar Experimentos.
\begin{table}[p]
	\centering
	\begin{tabularx}{\linewidth}{ p{0.21\columnwidth} p{0.71\columnwidth} }
		\toprule
		\textbf{CU-1}    & \textbf{Ejemplo de caso de uso}\\
		\toprule
		\textbf{Versión}              & 1.0    \\
		\textbf{Autor}                & Alumno \\
		\textbf{Requisitos asociados} & RF-xx, RF-xx \\
		\textbf{Descripción}          & La descripción del CU \\
		\textbf{Precondición}         & Precondiciones (podría haber más de una) \\
		\textbf{Acciones}             &
		\begin{enumerate}
			\def\labelenumi{\arabic{enumi}.}
			\tightlist
			\item Pasos del CU
			\item Pasos del CU (añadir tantos como sean necesarios)
		\end{enumerate}\\
		\textbf{Postcondición}        & Postcondiciones (podría haber más de una) \\
		\textbf{Excepciones}          & Excepciones \\
		\textbf{Importancia}          & Alta o Media o Baja... \\
		\bottomrule
	\end{tabularx}
	\caption{CU-1 Nombre del caso de uso.}
\end{table}

%%%%%%%%%%%%%%%%%%%%%%%%%%%%%%%%%%%%%%%%%%%%%%%%%%%%%%%%%%%%%%%%

\section{Objetivos generales}

%%%%%%%%%%%%%%%%%%%%%%%%%%%%%%%%%%%%%%%%%%%%%%%%%%%%%%%%%%%%%%%%

\section{Catálogo de requisitos}

\subsection{Requisitos funcionales}

\todo[inline]{Pendiente añadir descripción breve a cada uno.}

\begin{itemize}
	\item \textbf{RF-1 Gestión de productos:} ... 
	\begin{itemize}
		\item \textbf{RF-1.1 Agregar o borrar un producto:} ...
		\item \textbf{RF-1.2 Modificar información de un producto:} ...
		\item \textbf{RF-1.3 Visualizar información nutricional:} ...
	\end{itemize}
	
	\item \textbf{RF-2 Gestión de pedidos:} tanto en barra como en comedor...
	\begin{itemize}
		\item \textbf{RF-2.1 Creación de nuevos pedidos:} ...
		\begin{itemize}
			\item \textbf{RF-2.1.1 Añadir nuevos pedidos:} ...
			\item \textbf{RF-2.1.2 Incrementar/decrementar cantidades pedidas:} ...
			\item \textbf{RF-2.1.3 Actualización automática del precio total del pedido:} ...
		\end{itemize}
		\item \textbf{RF-2.2 Visualización de los pedidos existentes:} ...
		\begin{itemize}
			\item \textbf{RF-2.2.1 Modificación de los pedidos:} ...
			\item \textbf{RF-2.2.2 Cancelación de los pedidos:} ...
		\end{itemize}
		\item \textbf{RF-2.3 Modificación de las raciones ofrecidas:} ...
		\begin{itemize}
			\item \textbf{RF-2.3.1 Añadir/borrar productos:} ...
			\item \textbf{RF-2.3.2 Modificar precio unitario de producto:} ...
		\end{itemize}
	\end{itemize}
	
	\item \textbf{RF-3 Gestión de reservas de mesas:} ...
	\begin{itemize}
		\item \textbf{RF-3.1 Visualización de mesas sin reservar:} ...
		\begin{itemize}
			\item \textbf{RF-3.1.1 Asignación de nuevos comensales a mesas:} ...
		\end{itemize}
		\item \textbf{RF-3.2 Visualización de mesas reservadas:} ...
		\begin{itemize}
			\item \textbf{RF-3.2.1 Visualización del número de mesa y hora de reserva:} ...
			\item \textbf{RF-3.2.2 Modificación de la reserva de la mesa:} ...
			\item \textbf{RF-3.2.3 Cancelación de la reserva de la mesa:} ...
		\end{itemize}
		\item \textbf{RF-3.3 Modificación de las mesas por administrador:} tamaño, cantidad...
	\end{itemize}
	
	\item \textbf{RF-4 Gestión de usuarios:} ...
	\begin{itemize}
		\item \textbf{RF-4.1 Identificación del usuario:} ...
		\begin{itemize}
			\item \textbf{RF-4.1.1 Búsqueda del usuario en el sistema:} ...
			\item \textbf{RF-4.1.2 Validación de contraseña:} ...
		\end{itemize}
		\item \textbf{RF-4.2 Información de los usuarios :} ...
		\begin{itemize}
			\item \textbf{RF-4.2.1 Añadir o borrar un usuario:} ...
			\item \textbf{RF-4.2.2 Modificación parámetros de un usuario:} ...
		\end{itemize}
	\end{itemize}
\end{itemize}

%%%%%%%%%%%%%%%%%%%%%%%%%%%%%%%%%%%%%%%%%%%%%%%%%%%%%%%%%%%%%%%%

\section{Especificación de requisitos}


