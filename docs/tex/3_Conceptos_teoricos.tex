\capitulo{3}{Conceptos teóricos}

\section{\textit{Nutri-Score}}
El \textit{Nutri-Score} es un sistema de etiquetado frontal que clasifica la calidad nutricional de los alimentos en una escala de cinco clases representadas por letras y colores, desde la ``\textbf{A}'' (verde) hasta la ``\textbf{E}'' (rojo), siendo ``A'' la mejor calidad nutricional~\cite{santePubliqueFrance2025:nutriscore}. Su propósito es ofrecer al consumidor una valoración rápida y comparativa del perfil nutricional de un producto, facilitando decisiones alimentarias informadas. 

\imagenConEtiqueta[0.8\textwidth]{img/apdo_3/3_nutriscore.png}{Etiquetas A-E según \textit{Nutri-Score}}{3_nutriscore}

El resultado numérico (\textit{score}) se obtiene por un sistema de puntos que combina ``puntos negativos'' y ``puntos positivos'', calculados por 100 g o 100 ml del producto. Los puntos negativos penalizan contenidos elevados de energía (kJ), azúcares disponibles, ácidos grasos saturados y sodio, mientras que los puntos positivos recompensan la presencia de fruta/verdura/frutos secos/legumbres, fibra y proteínas. El \textit{score} final se calcula restando los puntos positivos de los negativos y se asigna a una letra A–E mediante umbrales que pueden variar según la categoría del alimento (p. ej. bebidas, quesos o grasas añadidas), existiendo adaptaciones específicas para dichas familias. Para el cálculo exacto y las adaptaciones más recientes, existen calculadoras y documentación técnica pública~\cite{santePubliqueFrance2025:nutriscoreCalc}.

\section{Clasificación NOVA}
La clasificación NOVA organiza los alimentos según el grado y el propósito del procesamiento industrial, en lugar de centrarse únicamente en su composición nutricional.

NOVA pretende capturar efectos asociados al grado de procesamiento (por ejemplo patrones de consumo y riesgos poblacionales) y no reemplaza otros índices nutricionales. Por ejemplo, un producto ultraprocesado puede tener un perfil nutricional ``mejor'' según ciertos nutrientes, pero sigue perteneciendo al Grupo 4 por su naturaleza y propósito tecnológico.

\imagenConEtiqueta[0.45\textwidth]{img/apdo_3/3_nova.png}{Etiquetas 1-4 según clasificación NOVA}{3_nova}

La clasificación consta de cuatro grupos principales~\cite{monteiro2019:nova}:
\begin{enumerate}
	\item \textbf{Grupo 1 - Alimentos sin procesar o mínimamente procesados:} alimentos comestibles en su forma natural o sometidos a procesos simples (lavado, corte, pasteurización).
	\item \textbf{Grupo 2 - Ingredientes culinarios procesados:} sustancias extraídas o refinadas de alimentos (aceites, mantequillas, harinas, azúcares) destinadas a cocinar o condimentar.
	\item \textbf{Grupo 3 - Alimentos procesados:} productos sencillos elaborados con ingredientes del grupo 1 y 2 (conservas, quesos, panes simples).
	\item \textbf{Grupo 4 - Alimentos ultraprocesados (UPF\footnote{\textit{Ultra-Processed Foods}}):} formulaciones industriales con numerosos ingredientes (azúcares, aceites refinados, aditivos, sustancias poco usadas en la cocina doméstica) y procesos tecnológicos que tienden a producir productos agradables al gusto, de larga duración y altamente transformados. 
\end{enumerate}

\section{Regulación de alérgenos (Reglamento UE 1169/2011)}

En la Unión Europea, la normativa sobre información alimentaria al consumidor (Reglamento (UE) nº 1169/2011~\cite{reglamento1169_2011}) establece una lista de \textbf{14 alérgenos} cuya presencia en ingredientes debe identificarse de forma clara en el etiquetado de los alimentos destinados al consumidor final. Esta obligación se aplica tanto a los ingredientes utilizados como tal, como a los derivados que contengan alérgenos, y exige que los alérgenos sean fácilmente identificables en la lista de ingredientes (por ejemplo, destacándolos en negrita).

\vspace{-1ex}
\imagenConEtiqueta[0.75\textwidth]{img/apdo_3/3_alergenos.png}{Catorce alérgenos de declaración obligatoria de la UE}{3_alergenos}

La lista, en formato resumido, de ingredientes que deben ser declarados como alérgenos es:
\begin{enumerate}
	\setlength{\itemsep}{1pt}
	\setlength{\parskip}{0pt}
	\setlength{\parsep}{0pt}
		\item Cereales que contienen gluten (p. ej. trigo, centeno, cebada, avena).
		\item Crustáceos.
		\item Huevos.
		\item Pescado.
		\item Cacahuete.
		\item Soja.
		\item Leche (incluida la lactosa).
		\item Frutos de cáscara (nueces, almendras, avellanas, etc.).
		\item Apio.
		\item Mostaza.
		\item Sésamo (semillas de sésamo).
		\item Dióxido de azufre y sulfitos (si su concentración es $> 10$ mg/kg o 10 mg/l expresados como SO$_2$).
		\item Altramuces.
		\item Moluscos.
\end{enumerate}

