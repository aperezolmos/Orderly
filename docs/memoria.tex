\documentclass[a4paper,12pt,twoside]{memoir}

%%%%%%%%%%%%%%%%%%%%%%%%%%%%%%
\usepackage{todonotes}
\setuptodonotes{fancyline, color=blue!15}

\usepackage{tabularx}
\usepackage{array}
%%%%%%%%%%%%%%%%%%%%%%%%%%%%%%

% Castellano
\usepackage[spanish,es-tabla]{babel}
\selectlanguage{spanish}
\usepackage[utf8]{inputenc}
\usepackage[T1]{fontenc}
\usepackage{lmodern} % Scalable font
\usepackage{microtype}
\usepackage{placeins}

\RequirePackage{booktabs}
\RequirePackage[table]{xcolor}
\RequirePackage{xtab}
\RequirePackage{multirow}

% Links
\PassOptionsToPackage{hyphens}{url}\usepackage[colorlinks]{hyperref}
\hypersetup{
	allcolors = {red}
}

% Ecuaciones
\usepackage{amsmath}

% Rutas de fichero / paquete
\newcommand{\ruta}[1]{{\sffamily #1}}

% Párrafos
\nonzeroparskip

% Huérfanas y viudas
\widowpenalty100000
\clubpenalty100000

% Imágenes

% Comando para insertar una imagen en un lugar concreto.
% Los parámetros son:
% 1 --> Ruta absoluta/relativa de la figura
% 2 --> Texto a pie de figura
% 3 --> Tamaño en tanto por uno relativo al ancho de página
\usepackage{graphicx}
\newcommand{\imagen}[3]{
	\begin{figure}[!h]
		\centering
		\includegraphics[width=#3\textwidth]{#1}
		\caption{#2}\label{fig:#1}
	\end{figure}
	\FloatBarrier
}

% Comando para insertar una imagen sin posición.
% Los parámetros son:
% 1 --> Ruta absoluta/relativa de la figura
% 2 --> Texto a pie de figura
% 3 --> Tamaño en tanto por uno relativo al ancho de página
\newcommand{\imagenflotante}[3]{
	\begin{figure}
		\centering
		\includegraphics[width=#3\textwidth]{#1}
		\caption{#2}\label{fig:#1}
	\end{figure}
}

% El comando \figura nos permite insertar figuras comodamente, y utilizando
% siempre el mismo formato. Los parametros son:
% 1 --> Porcentaje del ancho de página que ocupará la figura (de 0 a 1)
% 2 --> Fichero de la imagen
% 3 --> Texto a pie de imagen
% 4 --> Etiqueta (label) para referencias
% 5 --> Opciones que queramos pasarle al \includegraphics
% 6 --> Opciones de posicionamiento a pasarle a \begin{figure}
\newcommand{\figuraConPosicion}[6]{%
  \setlength{\anchoFloat}{#1\textwidth}%
  \addtolength{\anchoFloat}{-4\fboxsep}%
  \setlength{\anchoFigura}{\anchoFloat}%
  \begin{figure}[#6]
    \begin{center}%
      \Ovalbox{%
        \begin{minipage}{\anchoFloat}%
          \begin{center}%
            \includegraphics[width=\anchoFigura,#5]{#2}%
            \caption{#3}%
            \label{#4}%
          \end{center}%
        \end{minipage}
      }%
    \end{center}%
  \end{figure}%
}

%
% Comando para incluir imágenes en formato apaisado (sin marco).
\newcommand{\figuraApaisadaSinMarco}[5]{%
  \begin{figure}%
    \begin{center}%
    \includegraphics[angle=90,height=#1\textheight,#5]{#2}%
    \caption{#3}%
    \label{#4}%
    \end{center}%
  \end{figure}%
}
% Para las tablas
\newcommand{\otoprule}{\midrule [\heavyrulewidth]}
%
% Nuevo comando para tablas pequeñas (menos de una página).
\newcommand{\tablaSmall}[5]{%
 \begin{table}
  \begin{center}
   \rowcolors {2}{gray!35}{}
   \begin{tabular}{#2}
    \toprule
    #4
    \otoprule
    #5
    \bottomrule
   \end{tabular}
   \caption{#1}
   \label{tabla:#3}
  \end{center}
 \end{table}
}

%
% Nuevo comando para tablas pequeñas (menos de una página).
\newcommand{\tablaSmallSinColores}[5]{%
 \begin{table}[H]
  \begin{center}
   \begin{tabular}{#2}
    \toprule
    #4
    \otoprule
    #5
    \bottomrule
   \end{tabular}
   \caption{#1}
   \label{tabla:#3}
  \end{center}
 \end{table}
}

\newcommand{\tablaApaisadaSmall}[5]{%
\begin{landscape}
  \begin{table}
   \begin{center}
    \rowcolors {2}{gray!35}{}
    \begin{tabular}{#2}
     \toprule
     #4
     \otoprule
     #5
     \bottomrule
    \end{tabular}
    \caption{#1}
    \label{tabla:#3}
   \end{center}
  \end{table}
\end{landscape}
}

%
% Nuevo comando para tablas grandes con cabecera y filas alternas coloreadas en gris.
\newcommand{\tabla}[6]{%
  \begin{center}
    \tablefirsthead{
      \toprule
      #5
      \otoprule
    }
    \tablehead{
      \multicolumn{#3}{l}{\small\sl continúa desde la página anterior}\\
      \toprule
      #5
      \otoprule
    }
    \tabletail{
      \hline
      \multicolumn{#3}{r}{\small\sl continúa en la página siguiente}\\
    }
    \tablelasttail{
      \hline
    }
    \bottomcaption{#1}
    \rowcolors {2}{gray!35}{}
    \begin{xtabular}{#2}
      #6
      \bottomrule
    \end{xtabular}
    \label{tabla:#4}
  \end{center}
}

%
% Nuevo comando para tablas grandes con cabecera.
\newcommand{\tablaSinColores}[6]{%
  \begin{center}
    \tablefirsthead{
      \toprule
      #5
      \otoprule
    }
    \tablehead{
      \multicolumn{#3}{l}{\small\sl continúa desde la página anterior}\\
      \toprule
      #5
      \otoprule
    }
    \tabletail{
      \hline
      \multicolumn{#3}{r}{\small\sl continúa en la página siguiente}\\
    }
    \tablelasttail{
      \hline
    }
    \bottomcaption{#1}
    \begin{xtabular}{#2}
      #6
      \bottomrule
    \end{xtabular}
    \label{tabla:#4}
  \end{center}
}

%
% Nuevo comando para tablas grandes sin cabecera.
\newcommand{\tablaSinCabecera}[5]{%
  \begin{center}
    \tablefirsthead{
      \toprule
    }
    \tablehead{
      \multicolumn{#3}{l}{\small\sl continúa desde la página anterior}\\
      \hline
    }
    \tabletail{
      \hline
      \multicolumn{#3}{r}{\small\sl continúa en la página siguiente}\\
    }
    \tablelasttail{
      \hline
    }
    \bottomcaption{#1}
  \begin{xtabular}{#2}
    #5
   \bottomrule
  \end{xtabular}
  \label{tabla:#4}
  \end{center}
}



\definecolor{cgoLight}{HTML}{EEEEEE}
\definecolor{cgoExtralight}{HTML}{FFFFFF}

%
% Nuevo comando para tablas grandes sin cabecera.
\newcommand{\tablaSinCabeceraConBandas}[5]{%
  \begin{center}
    \tablefirsthead{
      \toprule
    }
    \tablehead{
      \multicolumn{#3}{l}{\small\sl continúa desde la página anterior}\\
      \hline
    }
    \tabletail{
      \hline
      \multicolumn{#3}{r}{\small\sl continúa en la página siguiente}\\
    }
    \tablelasttail{
      \hline
    }
    \bottomcaption{#1}
    \rowcolors[]{1}{cgoExtralight}{cgoLight}

  \begin{xtabular}{#2}
    #5
   \bottomrule
  \end{xtabular}
  \label{tabla:#4}
  \end{center}
}



%%%%%%%%%%%%%%%%%%%%%%%%%%%%%%
% Comandos añadidos por mí

\newcommand{\revisionA}[1]{\colorbox{red!40}{#1}}
\newcommand{\revisionB}[1]{\colorbox{yellow!75}{#1}}
\newcommand{\revisionC}[1]{\colorbox{green!60}{#1}}


% El otro comando de imagen genera la etiqueta a partir de la ruta
\newcommand{\imagenConEtiqueta}[4][0.9\textwidth]{
	\begin{figure}[ht]
		\centering
		\includegraphics[width=#1]{#2}  % #1 es el tamaño (opcional) | #2 es la RUTA
		\caption{#3}\label{fig:#4}      % #4 es la ETIQUETA LIMPIA
	\end{figure}
	\FloatBarrier
}


% TABLA COMPARATIVA (apartado 6)
% Definición de colores
\definecolor{verdeSuave}{RGB}{208, 250, 180}
\definecolor{rojoSuave}{RGB}{255, 205, 210}
\definecolor{amarilloSuave}{RGB}{252, 252, 162}

% Comandos con alineación y color
\newcommand{\SI}{\cellcolor{verdeSuave}Sí}
\newcommand{\NO}{\cellcolor{rojoSuave}No}
\newcommand{\PARCIAL}{\cellcolor{amarilloSuave}Parcial}

% Definición de columnas
\renewcommand{\tabularxcolumn}[1]{m{#1}}
\newcolumntype{C}{>{\centering\arraybackslash}X}
\newcolumntype{L}{>{\raggedright\arraybackslash}p{3.4cm}}
%%%%%%%%%%%%%%%%%%%%%%%%%%%%%%



\graphicspath{ {./img/} }

% Capítulos
\chapterstyle{bianchi}
\newcommand{\capitulo}[2]{
	\setcounter{chapter}{#1}
	\setcounter{section}{0}
	\setcounter{figure}{0}
	\setcounter{table}{0}
	\chapter*{\thechapter.\enskip #2}
	\addcontentsline{toc}{chapter}{\thechapter.\enskip #2}
	\markboth{#2}{#2}
}

% Apéndices
\renewcommand{\appendixname}{Apéndice}
\renewcommand*\cftappendixname{\appendixname}

\newcommand{\apendice}[1]{
	%\renewcommand{\thechapter}{A}
	\chapter{#1}
}

\renewcommand*\cftappendixname{\appendixname\ }

% Formato de portada
\makeatletter
\usepackage{xcolor}
\newcommand{\tutor}[1]{\def\@tutor{#1}}
\newcommand{\course}[1]{\def\@course{#1}}
\definecolor{cpardoBox}{HTML}{E6E6FF}
\def\maketitle{
  \null
  \thispagestyle{empty}
  % Cabecera ----------------
\noindent\includegraphics[width=\textwidth]{cabecera}\vspace{1cm}%
  \vfill
  % Título proyecto y escudo informática ----------------
  \colorbox{cpardoBox}{%
    \begin{minipage}{.8\textwidth}
      \vspace{.5cm}\Large
      \begin{center}
      \textbf{TFG del Grado en Ingeniería Informática}\vspace{.6cm}\\
      \textbf{\LARGE\@title{}}
      \end{center}
      \vspace{.2cm}
    \end{minipage}

  }%
  \hfill\begin{minipage}{.20\textwidth}
    \includegraphics[width=\textwidth]{escudoInfor}
  \end{minipage}
  \vfill
  % Datos de alumno, curso y tutores ------------------
  \begin{center}%
  {%
    \noindent\LARGE
    Presentado por \@author{}\\ 
    en Universidad de Burgos --- \@date{}\\
    Tutor: \@tutor{}\\
  }%
  \end{center}%
  \null
  \cleardoublepage
  }
\makeatother

\newcommand{\nombre}{Amanda Pérez Olmos} %%% cambio de comando

% Datos de portada
\title{Aplicación web para la gestión de negocios de restauración con soporte de datos nutricionales: Orderly}
\author{\nombre}
\tutor{Raúl Marticorena Sánchez}
\date{\today}

\begin{document}

\maketitle


\newpage\null\thispagestyle{empty}\newpage


%%%%%%%%%%%%%%%%%%%%%%%%%%%%%%%%%%%%%%%%%%%%%%%%%%%%%%%%%%%%%%%%%%%%%%%%%%%%%%%%%%%%%%%%
\thispagestyle{empty}


\noindent\includegraphics[width=\textwidth]{cabecera}\vspace{1cm}

\noindent D. Raúl Marticorena Sánchez, profesor del departamento de Ingeniería Informática, área de Lenguajes y Sistemas Informáticos.

\noindent Expone:

\noindent Que la alumna Dña. \nombre, con DNI 71365665T, ha realizado el Trabajo final de Grado en Ingeniería Informática titulado ``Aplicación web para la gestión de negocios de restauración con soporte de datos nutricionales: Orderly''. 

\noindent Y que dicho trabajo ha sido realizado por la alumna bajo la dirección del que suscribe, en virtud de lo cual se autoriza su presentación y defensa.

\begin{center} %\large
En Burgos, {\large \today}
\end{center}

\vfill\vfill\vfill

% Author and supervisor
%\begin{minipage}{0.45\textwidth}
%\begin{flushleft} %\large
%Vº. Bº. del Tutor:\\[2cm]
%D. Raúl Marticorena Sánchez
%\end{flushleft}
%\end{minipage}
\hfill
% \begin{minipage}{0.45\textwidth}
	% \begin{flushleft} %\large
		% Vº. Bº. del co-tutor:\\[2cm]
		% D. nombre co-tutor
		% \end{flushleft}
	% \end{minipage}
% \hfill

\vfill

% para casos con solo un tutor comentar lo anterior
% y descomentar lo siguiente
%Vº. Bº. del Tutor:\\[2cm]
%D. Raúl Marticorena Sánchez

V.{\textsuperscript{o}} B.{\textsuperscript{o}} del Tutor:\\[2cm]

D. Raúl Marticorena Sánchez

\newpage\null\thispagestyle{empty}\newpage




\frontmatter

% Abstract en castellano
\renewcommand*\abstractname{Resumen}
\begin{abstract}
\textit{Orderly} es una aplicación web integral diseñada para optimizar y centralizar la gestión operativa de locales de restauración (bares, restaurantes, cafeterías). Su objetivo principal es unificar en una única plataforma los flujos de trabajo fundamentales de este sector: la gestión de productos y su composición, el control de comandas (pedidos de bar y comedor), la administración de reservas de mesas y la gestión de usuarios y permisos del personal.

Una característica diferenciadora de \textit{Orderly} es su integración nativa con \textit{\textbf{Open Food Facts}}, una base de datos colaborativa de información nutricional. Esta funcionalidad permite enriquecer el catálogo de alimentos con datos nutricionales detallados, clasificaciones como el \textit{\textbf{Nutri-Score}} y la identificación de alérgenos de forma semi-automática, respondiendo así a la creciente demanda de transparencia y opciones de alimentación saludable por parte de los consumidores.

El desarrollo se ha abordado con un fuerte énfasis en la calidad del \textit{software}, la escalabilidad y la mantenibilidad. Se ha implementado una arquitectura moderna basada en \textit{\textbf{React}} para el \textit{frontend} y \textbf{\textit{Spring Boot}} para el \textit{backend} (API REST), utilizando contenedores \textit{\textbf{Docker}} para garantizar un despliegue portable y reproducible. La metodología de trabajo ha seguido prácticas ágiles (\textit{Scrum/Kanban}), con un riguroso control de versiones mediante \textit{Git} y \textit{\textbf{GitHub}}.
\end{abstract}

\renewcommand*\abstractname{Descriptores}
\begin{abstract}
Gestión de restauración, información nutricional, Nutri-Score, alérgenos, aplicación web, React, Spring Boot, API REST, Open Food Facts, Docker, contenerización, desarrollo ágil.
\end{abstract}

\clearpage

% Abstract en inglés
\renewcommand*\abstractname{Abstract}
\begin{abstract}
\textit{Orderly} is a comprehensive web application designed to optimise and centralise the operational management of hospitality venues (bars, restaurants, cafés). Its main goal is to unify the fundamental workflows of this sector into a single platform: product and recipe management, order control (for both bar and dining areas), table reservation administration, and staff user and permission management.

A key distinguishing feature of \textit{Orderly} is its native integration with \textit{\textbf{Open Food Facts}}, a collaborative database of nutritional information. This functionality enriches the food catalogue with detailed nutritional data, classifications such as \textit{\textbf{Nutri-Score}}, and semi-automatic allergen identification, addressing the growing consumer demand for transparency and healthy food options.

The development has been approached with a strong emphasis on software quality, scalability, and maintainability. A modern architecture based on \textit{\textbf{React}} for the frontend and \textit{\textbf{Spring Boot}} for the backend (REST API) has been implemented, using \textit{\textbf{Docker}} containers to ensure portable and reproducible deployment. The workflow followed agile practices (\textit{Scrum/Kanban}), with rigorous version control using \textit{Git} and \textit{\textbf{GitHub}}.
\end{abstract}

\renewcommand*\abstractname{Keywords}
\begin{abstract}
Restaurant management, nutritional information, Nutri-Score, allergens, web application, React, Spring Boot, REST API, Open Food Facts, Docker, containerization, agile development.
\end{abstract}

\clearpage

% Indices
\tableofcontents

\clearpage

\listoffigures

\clearpage

\listoftables
\clearpage

\mainmatter
\capitulo{1}{Introducción}

El sector de la restauración es un entorno dinámico y complejo donde la eficiencia operativa es un factor clave para el éxito. Tradicionalmente, la gestión de estos establecimientos ha dependido de una combinación de herramientas dispersas (desde registros manuales hasta software especializado a menudo cerrado y poco adaptable) que no siempre se comunican entre sí. Esta fragmentación de procesos aumenta la complejidad operativa y el margen de error, dificultando el flujo de trabajo entre cocina y sala.

En este contexto, \textit{Orderly} surge como un proyecto con el objetivo de desarrollar una solución web integral, moderna y centralizada. La aplicación busca ser un \textbf{punto de gestión unificado} que cubra los procesos esenciales de un local: desde la creación y composición de platos, pasando por el registro de comandas y la gestión de reservas, hasta la administración de los roles del equipo. Al integrar estos módulos en una plataforma modular y extensible, se pretende no solo agilizar los flujos de trabajo, sino ofrecer una herramienta que se adapte con facilidad a distintos tipos de establecimientos y necesidades.

Un pilar diferenciador de este proyecto es su compromiso con la \textbf{transparencia nutricional}. La integración directa con la API de \textit{\textbf{Open Food Facts}} permite a los establecimientos automatizar la consulta de ingredientes y alérgenos, facilitando la elaboración de fichas de productos precisas y verificadas. En una era donde la conciencia sobre la alimentación saludable es creciente, esta funcionalidad añade un valor significativo y contemporáneo a la herramienta.

Este documento constituye la \textbf{memoria} principal del proyecto, donde se detallan los objetivos, conceptos teóricos, técnicas empleadas y los aspectos más relevantes del desarrollo. La documentación técnica completa, que incluye la planificación detallada del proyecto, la especificación de requisitos y diseño, así como los manuales de programador y usuario, se presenta en un conjunto de \textbf{anexos} independientes.

Todo el código fuente, la documentación y el histórico de desarrollo están disponibles públicamente en el repositorio de \textit{GitHub} del proyecto:

{\centering \url{https://github.com/aperezolmos/Orderly} \par}

El progreso del proyecto ha sido gestionado mediante la herramienta \textit{GitHub Projects}, cuyo tablero y gráficas asociadas pueden visualizarse en:

{\centering \url{https://github.com/users/aperezolmos/projects/4} \par}

El reporte de calidad del código realizado por \textit{SonarCloud} puede visualizarse en:\todo{Aún está privado}

{\centering \url{https://sonarcloud.io/project/overview?id=aperezolmos_Orderly} \par}

\capitulo{2}{Objetivos del proyecto}

A continuación, se detallan los objetivos que han guiado el desarrollo del proyecto.

\section{Objetivos generales}
\begin{enumerate}
%	\setlength{\itemsep}{1pt}
%	\setlength{\parskip}{0pt}
%	\setlength{\parsep}{0pt}
	\item \textbf{Crear un sistema de gestión centralizado para locales de restauración} que unifique las operaciones críticas de gestión de productos, pedidos, reservas y usuarios en una única interfaz web.
	\item \textbf{Integrar información nutricional externa} de manera automática, mediante el consumo de una API pública, para añadir valor diferencial y promover la transparencia alimentaria.
	\item \textbf{Facilitar la identificación y gestión de alérgenos}, permitiendo asociarlos a los productos disponibles y ofrecer mecanismos de filtrado que ayuden al personal a identificar opciones seguras.
	\item\textbf{Demostrar competencia en el ciclo completo de desarrollo de \textit{software}}, desde el análisis y diseño hasta la implementación, pruebas, despliegue y documentación, aplicando los conocimientos adquiridos durante el grado.
\end{enumerate}

\section{Objetivos técnicos}

\begin{enumerate}
%	\setlength{\itemsep}{1pt}
%	\setlength{\parskip}{0pt}
%	\setlength{\parsep}{0pt}
	\item \textbf{Implementar una arquitectura cliente-servidor desacoplada}: desarrollar un \textit{frontend} dinámico con \textit{React} que consuma una API REST construida con \textit{Spring Boot}.
	\item \textbf{Diseñar un modelo de datos relacional eficiente y mantenible}: utilizar \textit{Spring Data JPA} sobre \textit{MySQL} para definir un esquema de base de datos normalizado, incorporando un modelo de dominio enriquecido que encapsule lógica de negocio.
	\item \textbf{Diseñar una jerarquía de pedidos escalable}: emplear estrategias de herencia en JPA que optimicen las consultas y faciliten la extensión del modelo.
	\item \textbf{Separar el modelo interno de la interfaz expuesta}: estructurar la comunicación API mediante DTOs y \textit{mappers} automatizados de \textit{MapStruct} que aseguren conversiones eficientes.
	\item \textbf{Implementar un sistema de seguridad robusto}: configurar \textit{Spring Security} para gestionar la autenticación y una autorización granular, utilizando un modelo de roles y permisos que permita un control de acceso detallado a los recursos de la API.
	\item \textbf{Asegurar la calidad y robustez del \textit{backend}}: implementar un sistema completo de validación de entradas, manejo centralizado de excepciones y un conjunto de pruebas automatizadas (unitarias y de integración) para los componentes clave de la API (servicios, \textit{mappers} y lógica de negocio).
	\item \textbf{Establecer un flujo básico de Integración Continua (CI)}: configurar un \textit{workflow} con \textit{GitHub Actions} para ejecutar automáticamente la \textit{suite} de pruebas del \textit{backend} con cada \textit{commit}, mejorando la detección temprana de errores y la confiabilidad del código base.
	\item \textbf{Construir una interfaz de usuario moderna, reactiva e internacionalizable}: desarrollar el \textit{frontend} con componentes reutilizables, gestionar el estado de la aplicación de forma eficiente e implementar el soporte a varios idiomas (español/inglés).
	\item \textbf{Garantizar la portabilidad y reproducibilidad del despliegue}: utilizar \textit{Docker} para \textit{contenerizar} todos los servicios de la aplicación (base de datos, \textit{backend} y \textit{frontend}), creando un entorno de ejecución consistente e independiente de la infraestructura subyacente.
	\item \textbf{Aplicar metodologías ágiles y control de versiones}: gestionar el proyecto mediante \textit{sprints}, utilizando \textit{Git} con \textit{GitHub Flow} para mantener un historial de cambios claro, trazable y semántico.
\end{enumerate}

\capitulo{3}{Conceptos teóricos}

\section{\textit{Nutri-Score}}
El \textit{Nutri-Score} es un sistema de etiquetado frontal que clasifica la calidad nutricional de los alimentos en una escala de cinco clases representadas por letras y colores, desde la ``\textbf{A}'' (verde) hasta la ``\textbf{E}'' (rojo), siendo ``A'' la mejor calidad nutricional~\cite{santePubliqueFrance2025:nutriscore}. Su propósito es ofrecer al consumidor una valoración rápida y comparativa del perfil nutricional de un producto, facilitando decisiones alimentarias informadas. 

\imagenConEtiqueta[0.8\textwidth]{img/apdo_3/3_nutriscore.png}{Etiquetas A-E según \textit{Nutri-Score}}{3_nutriscore}

El resultado numérico (\textit{score}) se obtiene por un sistema de puntos que combina ``puntos negativos'' y ``puntos positivos'', calculados por 100 g o 100 ml del producto. Los puntos negativos penalizan contenidos elevados de energía (kJ), azúcares disponibles, ácidos grasos saturados y sodio, mientras que los puntos positivos recompensan la presencia de fruta/verdura/frutos secos/legumbres, fibra y proteínas. El \textit{score} final se calcula restando los puntos positivos de los negativos y se asigna a una letra A–E mediante umbrales que pueden variar según la categoría del alimento (p. ej. bebidas, quesos o grasas añadidas), existiendo adaptaciones específicas para dichas familias. Para el cálculo exacto y las adaptaciones más recientes, existen calculadoras y documentación técnica pública~\cite{santePubliqueFrance2025:nutriscoreCalc}.

\section{Clasificación NOVA}
La clasificación NOVA organiza los alimentos según el grado y el propósito del procesamiento industrial, en lugar de centrarse únicamente en su composición nutricional.

NOVA pretende capturar efectos asociados al grado de procesamiento (por ejemplo patrones de consumo y riesgos poblacionales) y no reemplaza otros índices nutricionales. Por ejemplo, un producto ultraprocesado puede tener un perfil nutricional ``mejor'' según ciertos nutrientes, pero sigue perteneciendo al Grupo 4 por su naturaleza y propósito tecnológico.

\imagenConEtiqueta[0.45\textwidth]{img/apdo_3/3_nova.png}{Etiquetas 1-4 según clasificación NOVA}{3_nova}

La clasificación consta de cuatro grupos principales~\cite{monteiro2019:nova}:
\begin{enumerate}
	\item \textbf{Grupo 1 - Alimentos sin procesar o mínimamente procesados:} alimentos comestibles en su forma natural o sometidos a procesos simples (lavado, corte, pasteurización).
	\item \textbf{Grupo 2 - Ingredientes culinarios procesados:} sustancias extraídas o refinadas de alimentos (aceites, mantequillas, harinas, azúcares) destinadas a cocinar o condimentar.
	\item \textbf{Grupo 3 - Alimentos procesados:} productos sencillos elaborados con ingredientes del grupo 1 y 2 (conservas, quesos, panes simples).
	\item \textbf{Grupo 4 - Alimentos ultraprocesados (UPF\footnote{\textit{Ultra-Processed Foods}}):} formulaciones industriales con numerosos ingredientes (azúcares, aceites refinados, aditivos, sustancias poco usadas en la cocina doméstica) y procesos tecnológicos que tienden a producir productos agradables al gusto, de larga duración y altamente transformados. 
\end{enumerate}

\section{Regulación de alérgenos (Reglamento UE 1169/2011)}

En la Unión Europea, la normativa sobre información alimentaria al consumidor (Reglamento (UE) nº 1169/2011~\cite{reglamento1169_2011}) establece una lista de \textbf{14 alérgenos} cuya presencia en ingredientes debe identificarse de forma clara en el etiquetado de los alimentos destinados al consumidor final. Esta obligación se aplica tanto a los ingredientes utilizados como tal, como a los derivados que contengan alérgenos, y exige que los alérgenos sean fácilmente identificables en la lista de ingredientes (por ejemplo, destacándolos en negrita).

\vspace{-1ex}
\imagenConEtiqueta[0.75\textwidth]{img/apdo_3/3_alergenos.png}{Catorce alérgenos de declaración obligatoria de la UE}{3_alergenos}

La lista, en formato resumido, de ingredientes que deben ser declarados como alérgenos es:
\begin{enumerate}
	\setlength{\itemsep}{1pt}
	\setlength{\parskip}{0pt}
	\setlength{\parsep}{0pt}
		\item Cereales que contienen gluten (p. ej. trigo, centeno, cebada, avena).
		\item Crustáceos.
		\item Huevos.
		\item Pescado.
		\item Cacahuete.
		\item Soja.
		\item Leche (incluida la lactosa).
		\item Frutos de cáscara (nueces, almendras, avellanas, etc.).
		\item Apio.
		\item Mostaza.
		\item Sésamo (semillas de sésamo).
		\item Dióxido de azufre y sulfitos (si su concentración es $> 10$ mg/kg o 10 mg/l expresados como SO$_2$).
		\item Altramuces.
		\item Moluscos.
\end{enumerate}


\capitulo{4}{Técnicas y herramientas}

En este capítulo se describen las metodologías y herramientas tecnológicas escogidas para el desarrollo del trabajo, detallando los aspectos principales y justificando los motivos de su selección.

%%%%%%%%%%%%%%%%%%%%%%%%%%%%%%%%%%%%%%%%%%%%%%%%%%%%%%%%%%%%%%%%

\section{Metodologías y técnicas de gestión}

\subsubsection{Scrum}
Scrum~\cite{atlassian:scrum} es un marco de trabajo ágil orientado a la gestión y desarrollo de proyectos complejos, especialmente en el ámbito del desarrollo de \textit{software}. Se basa en un enfoque iterativo e incremental, en el que el trabajo se organiza en ciclos de duración fija denominados \textit{sprints}. Durante cada \textit{sprint} se selecciona un conjunto de requisitos priorizados que deben ser diseñados, implementados y validados.

Scrum define una serie de 
\href{https://www.atlassian.com/es/agile/scrum/roles}{roles}, 
\href{https://www.atlassian.com/es/agile/scrum/ceremonies}{eventos} y 
\href{https://www.atlassian.com/es/agile/scrum/artifacts}{artefactos} 
que facilitan la planificación, la inspección continua del progreso y la adaptación a cambios en los requisitos, promoviendo la entrega frecuente de incrementos funcionales del producto.

En el marco de este trabajo, se adaptó esta metodología a las particularidades de un desarrollo académico individual. Para ello, se preservaron elementos fundamentales como la duración de los \textit{sprints} (1-2 semanas) y las reuniones de revisión tras finalizar cada iteración, destinadas a evaluar los avances logrados y planificar las siguientes mejoras.

\subsubsection{Kanban}
Kanban~\cite{atlassian:kanban} es una técnica de gestión visual del trabajo cuyo objetivo principal es optimizar el flujo de tareas y mejorar la eficiencia del proceso de desarrollo. Se basa en la representación gráfica del estado de las tareas mediante un tablero dividido en columnas que reflejan las distintas fases del proceso (p. ej. \textit{``In progress''} o \textit{``Done''}). En la figura \ref{fig:4_tablero_kanban} se puede observar el tablero Kanban utilizado durante el desarrollo de este proyecto. 

Cada tarea se representa como un elemento que se desplaza entre estados a medida que avanza su ejecución. Kanban pone especial énfasis en la limitación del trabajo en curso (\textit{Work In Progress}), la identificación de cuellos de botella y la mejora continua del proceso, sin imponer iteraciones temporales cerradas ni roles específicos.

\subsubsection{\textit{Rich Domain Model}}
El \textit{Rich Domain Model} o Modelo Enriquecido~\cite{memobackend:richDomain} es un enfoque de diseño propio de la arquitectura orientada a objetos y del \textit{Domain-Driven Design}~\cite{geeksforgeeks:DDD}, en el que las entidades del dominio encapsulan tanto los datos como el comportamiento asociado a los mismos.

A diferencia de los 
\href{https://memobackend.com.ar/2024-06-11-modelos-anemicos-enriquecidos/#:~:text=en%20las%20aplicaciones.-,Modelos%20An%C3%A9micos,negocio%20no%20est%C3%A1%20estrechamente%20ligada%20a%20los%20estados%20de%20los%20objetos.,-Modelos%20Enriquecidos%20o}{modelos anémicos}, 
en los que las entidades actúan únicamente como contenedores de información, un modelo de dominio enriquecido incluye lógica de negocio relevante dentro de las propias clases del dominio. Este enfoque favorece una mayor cohesión, una mejor representación del dominio del problema y una distribución más natural de las responsabilidades dentro del sistema.

\subsubsection{\textit{Domain-Driven Design} (\textit{DDD})}
El \textit{Domain-Driven Design}~\cite{geeksforgeeks:DDD} es un enfoque de diseño de \textit{software} orientado a la construcción de sistemas complejos a partir de un modelo de dominio rico y expresivo, alineado con el conocimiento del problema que se desea resolver. \textit{DDD} propone centrar el diseño en el dominio del negocio, fomentando una comunicación constante entre expertos del dominio y desarrolladores, utilizando conceptos como modelos de dominio, agregados, entidades, objetos de valor y contextos delimitados (\textit{bounded contexts}). El objetivo principal es lograr un diseño coherente, mantenible y adaptable a la evolución de los requisitos.

En el desarrollo de este proyecto se han adoptado algunos principios y conceptos inspirados en el \textit{Domain-Driven Design}, sin aplicar la metodología de forma estricta o completa. En particular, se ha priorizado una organización del código orientada a funcionalidades o dominios concretos del sistema (directorios \texttt{user}, \texttt{food}, \texttt{product}...), en lugar de una estructura clásica por capas (directorios \texttt{controller}, \texttt{service}, \texttt{repository}...). También se ha favorecido que las entidades del dominio encapsulen comportamientos y responsabilidades propias, especialmente en la gestión de sus relaciones. No obstante, dado el alcance y complejidad moderada del proyecto, no se ha llevado a cabo una implementación exhaustiva de todos los patrones y prácticas formales de \textit{DDD}, optando por una aplicación selectiva y funcional, adaptada al contexto del trabajo.

%%%%%%%%%%%%%%%%%%%%%%%%%%%%%%%%%%%%%%%%%%%%%%%%%%%%%%%%%%%%%%%%

\section{Patrón de arquitectura}
\label{sec:4.2_arquitectura}

La aplicación sigue una arquitectura \textbf{Cliente-Servidor} con un \textit{frontend} y un \textit{backend} claramente desacoplados. El \textit{frontend}, desarrollado en \textit{React}, se estructura en componentes reutilizables, páginas y servicios modulares que consumen la API mediante un cliente centralizado.

%\imagenConEtiqueta[0.55\textwidth]{img/apdo_4/4_arquitectura_capas.png}{Arquitectura en Capas~\cite{oreilly:architecture}}{4_arquitectura_capas}
%\vspace{-1em}

\imagenConEtiqueta[0.85\textwidth]{img/apen_C/C3_arquitectura.png}{Arquitectura de \textit{Orderly}}{4_arquitectura}

El \textit{backend} implementa una \textbf{Arquitectura en Capas} (\textit{Layered Architecture}) de cuatro niveles, la cual se puede visualizar en la figura \ref{fig:4_arquitectura}:
\vspace{-2em}
\begin{enumerate}
	\setlength{\itemsep}{0pt}
	\setlength{\parskip}{2pt}
%	\setlength{\parsep}{0pt}
		\item \textbf{Presentación:} controladores REST finos, responsables de manejar las peticiones HTTP y las respuestas.
		\item \textbf{Aplicación:} servicios encargados de la orquestación de operaciones y control transaccional.
		\item \textbf{Dominio:} entidades JPA con comportamiento (modelo rico) que se relacionan entre sí.
		\item \textbf{Persistencia:} repositorios JPA que abstraen \textit{MySQL}.
\end{enumerate}

%%%%%%%%%%%%%%%%%%%%%%%%%%%%%%%%%%%%%%%%%%%%%%%%%%%%%%%%%%%%%%%%

\section{Tecnologías y \textit{frameworks} principales}
\label{sec:4.3_tecnologias}

\subsection{\textit{Backend}}
\subsubsection{Spring Boot}
\textit{Spring Boot}~\cite{tokioschool:springBoot} es un \textit{framework} del ecosistema \textit{Spring}~\cite{spring:springFramework} diseñado para simplificar el desarrollo de aplicaciones \textit{backend} basadas en Java. Proporciona configuración automática, un sistema de dependencias gestionado y un servidor embebido, lo que permite crear aplicaciones listas para producción con un esfuerzo de configuración mínimo. Está especialmente orientado al desarrollo de servicios web y APIs REST.

\subsubsection{Spring Data JPA}
\textit{Spring Data JPA}~\cite{spring:springJPA} es un módulo de \textit{Spring}~\cite{spring:springFramework} que facilita el acceso y la persistencia de datos mediante la especificación
\href{https://www.ibm.com/docs/es/was-liberty/nd?topic=liberty-java-persistence-api-jpa}{JPA} 
(\textit{Java Persistence API}).

JPA define un estándar para el mapeo objeto-relacional (ORM), permitiendo representar tablas de bases de datos como entidades Java. \textit{Spring Data JPA} abstrae gran parte del código repetitivo asociado al acceso a datos, proporcionando repositorios y mecanismos automáticos de generación de consultas.

\subsubsection{Spring Security}
\textit{Spring Security}~\cite{spring:springSecurity} es un \textit{framework} orientado a la gestión de la seguridad en aplicaciones Java. Proporciona mecanismos para la autenticación, autorización y protección frente a ataques comunes o accesos no autorizados. Se integra de forma nativa con \textit{Spring Boot} y permite definir políticas de seguridad de manera flexible y extensible.

\subsubsection{Spring Validation}
\textit{Spring Validation}~\cite{spring:validation} es un módulo que permite validar datos de entrada mediante anotaciones declarativas. Se basa en la especificación 
\href{https://www.baeldung.com/java-validation}{\textit{Bean Validation}} 
y se utiliza habitualmente para comprobar la validez de los datos recibidos en peticiones, asegurando el cumplimiento de restricciones como valores obligatorios, rangos numéricos o formatos específicos.

En nuestra aplicación, los datos que se validan con estas anotaciones son DTOs (\textit{Data Transfer Objects}) de petición, es decir, los objetos que toman los controladores como valor de entrada. De esta manera, se pueden definir restricciones rápidas sobre cada campo para liberar a los servicios de comprobaciones exhaustivas.

\subsubsection{Lombok}
\textit{Lombok}~\cite{spring:lombok} es una biblioteca que reduce la cantidad de código repetitivo (\textit{``boilerplate''}) en aplicaciones Java mediante el uso de anotaciones. Permite generar automáticamente métodos comunes como \textit{getters}, \textit{setters}, constructores o métodos \texttt{equals} y \texttt{hashCode} durante el proceso de compilación, mejorando la legibilidad y mantenibilidad del código.

\subsubsection{MapStruct}
\textit{MapStruct}~\cite{spring:mapstruct} es una herramienta de mapeo de objetos que permite transformar de forma automática y segura objetos de un tipo a otro, como entidades y DTOs. Genera código en tiempo de compilación, lo que ofrece un alto rendimiento y evita errores en tiempo de ejecución, manteniendo una separación clara entre las distintas capas de la aplicación.

\subsubsection{MySQL}
\textit{MySQL}~\cite{wiki:mysql} es un sistema de gestión de bases de datos relacional ampliamente utilizado en aplicaciones empresariales y web. Se caracteriza por su rendimiento, fiabilidad y compatibilidad con múltiples plataformas. En el contexto de aplicaciones \textit{Spring Boot}, se emplea junto con el controlador JDBC correspondiente para permitir la comunicación entre la aplicación y la base de datos~\cite{spring:mysql}.

\subsection{\textit{Frontend}}
\subsubsection{React}
\textit{React}~\cite{react:intro} es una biblioteca de JavaScript para la construcción de interfaces de usuario basadas en componentes reutilizables. Se centra en la creación de vistas declarativas y en la gestión eficiente del estado y la actualización del \href{https://www.geeksforgeeks.org/javascript/dom-document-object-model/}{DOM} (\textit{Document Object Model}) 
mediante un modelo de renderizado reactivo.

Su elección se justifica frente a alternativas como \textit{Thymeleaf}~\cite{thymeleaf:tutorial} por las necesidades específicas del proyecto: la gestión de un elevado número de entidades (productos, pedidos, reservas, usuarios, etc.) y la exigencia de una interfaz dinámica y reactiva. El paradigma basado en \textbf{componentes} y la \textbf{arquitectura SPA} (\textit{Single Page Application}) de \textit{React} permiten un desarrollo más modular, mantenible y con mejor experiencia de usuario~\cite{wiki:spa}.

Pese a no haber utilizado \textit{React} previamente, el análisis del proyecto de referencia \textit{NutriMenu}~\cite{nutrimenu2024} permitió evaluar su viabilidad y agilizar el aprendizaje inicial. Su adopción supuso, por lo tanto, un reto formativo para dominar una herramienta estándar en el sector

\subsubsection{Vite}
\textit{Vite}~\cite{vite:intro} es una herramienta de construcción y desarrollo \textit{frontend} que proporciona un entorno de desarrollo rápido y optimizado. Utiliza un servidor de desarrollo basado en 
\href{https://lenguajejs.com/javascript/modulos/que-es-esm/}{módulos ES}
y genera \textit{builds} de producción altamente eficientes, reduciendo significativamente los tiempos de arranque y recarga.

\subsubsection{Mantine}
\textit{Mantine}~\cite{react:mantine} es una biblioteca de componentes para \textit{React} que proporciona elementos de interfaz modernos, accesibles y personalizables. Incluye componentes visuales, utilidades de estilo y \textit{hooks} que facilitan el desarrollo de interfaces coherentes y funcionales.

\subsubsection{React Router}
\textit{React Router}~\cite{react:router} es una biblioteca de enrutamiento para aplicaciones \textit{React} que permite gestionar la navegación entre distintas vistas sin recargar la página. Facilita la creación de aplicaciones de una sola página mediante rutas declarativas y dinámicas.

\subsubsection{i18next y react-i18next}
\textit{i18next}~\cite{react:i18next} es un \textit{framework} de internacionalización que permite gestionar traducciones y contenido multilingüe en aplicaciones \textit{frontend}. La librería \textit{react-i18next}~\cite{react:react-i18next} proporciona integración específica con \textit{React}, facilitando la adaptación dinámica de la interfaz al idioma seleccionado por el usuario.

\subsubsection{Recharts}
\textit{Recharts}~\cite{react:recharts} es una biblioteca de visualización de datos para \textit{React} basada en componentes reutilizables. Permite la creación de gráficos interactivos y personalizables a partir de datos estructurados, facilitando la representación visual de información compleja dentro de aplicaciones web.

Se ha utilizado esta librería para representar la información nutricional de los productos mediante gráficos interactivos, concretamente gráficos de tipo donut anidado. Estos gráficos permiten visualizar de forma intuitiva la distribución de los distintos valores nutricionales y mostrar información detallada al interactuar con ellos, como se observa en el ejemplo presente en la figura \ref{fig:4_recharts_grafico}.

\vspace{-1ex}
\imagenConEtiqueta[0.45\textwidth]{img/apdo_4/4_recharts_grafico.png}{Gráfico nutricional de un producto usando \textit{Recharts}}{4_recharts_grafico}

\subsubsection{Zustand}
\textit{Zustand}~\cite{react:zustand} es una biblioteca ligera para la gestión del estado global en aplicaciones \textit{React}. Se caracteriza por su simplicidad, bajo acoplamiento y uso de \textit{hooks}, permitiendo compartir estado entre componentes sin la complejidad de soluciones más pesadas.

\subsection{API de información nutricional: \textit{Open Food Facts}}

\textit{Open Food Facts} (\textit{OFF})~\cite{api:openFoodFacts} es una base de datos abierta y colaborativa que ofrece información detallada sobre productos alimentarios de todo el mundo a través de una API REST. La información proporcionada incluye datos generales del producto, información nutricional estandarizada, listas de ingredientes y la identificación de alérgenos, lo que la convierte en una fuente especialmente útil para aplicaciones del ámbito de la restauración y la nutrición.

Las principales \textbf{ventajas} que ofrece \textit{Open Food Facts} son las siguientes:
\begin{itemize}
	\setlength{\itemsep}{0pt}
	\item Amplio catálogo de productos, con millones de alimentos registrados y disponibilidad en múltiples idiomas, lo que facilita su uso en aplicaciones multilingües como la nuestra.
	\item Búsqueda mediante texto libre (\textit{full-text search}), que permite localizar productos a partir de su nombre, etiquetas, categorías o características nutricionales.
	\item Carácter abierto y acceso gratuito, sin restricciones estrictas en el número de peticiones a la API.
	\item Información detallada sobre alérgenos y métricas nutricionales. Este aspecto hace destacar a \textit{OFF} respecto al resto de opciones y fue decisivo para escogerla como API nutricional.
\end{itemize}

No obstante, el uso de \textit{Open Food Facts} también presenta algunas \textbf{limitaciones}:
\begin{itemize}
	\setlength{\itemsep}{0pt}
	\item Presencia de ruido en las búsquedas, debido al gran volumen y heterogeneidad de los datos, lo que puede dificultar la localización de productos concretos.
	\item Ausencia de alimentos genéricos o no envasados, como frutas, verduras u otros productos comunes que no disponen de código de barras.
	\item Inconsistencias en la completitud de la información, ya que, al tratarse de una base de datos colaborativa, algunos productos presentan campos incompletos o ausentes.
\end{itemize}

\subsection{\textit{Testing} y calidad del software}
\subsubsection{JUnit}
\textit{JUnit}~\cite{tutorialspoint:junit} es un \textit{framework} de pruebas unitarias para Java que permite verificar el correcto funcionamiento de componentes individuales del sistema. Facilita la automatización de pruebas y la detección temprana de errores durante el desarrollo.

\subsubsection{Mockito}
\textit{Mockito}~\cite{tutorialspoint:mockito} es una biblioteca de apoyo para pruebas que permite crear objetos simulados (\textit{mocks}) con el fin de aislar componentes y controlar su comportamiento durante la ejecución de las pruebas. Es especialmente útil para probar clases que dependen de otros componentes del sistema.

\subsubsection{Spring Boot Test}
\textit{Spring Boot Test} es un módulo que proporciona soporte específico para la realización de pruebas en aplicaciones \textit{Spring Boot}. Permite cargar contextos de aplicación, simular peticiones HTTP y realizar pruebas de integración de forma controlada~\cite{spring:integrationTest}.

\subsubsection{H2}
H2~\cite{medium:h2} es un sistema de base de datos relacional en memoria utilizado habitualmente en entornos de pruebas. Permite ejecutar pruebas de acceso a datos sin depender de una base de datos externa, garantizando rapidez y aislamiento en los tests.

\subsection{Contenedores y despliegue}
\subsubsection{Docker}
\textit{Docker}~\cite{docker:overview} es una plataforma de contenerización que permite empaquetar aplicaciones junto con sus dependencias en contenedores ligeros y portables. Esto garantiza que la aplicación se ejecute de forma consistente en diferentes entornos, facilitando el despliegue y la reproducibilidad.

\subsubsection{Docker Compose}
\textit{Docker Compose}~\cite{docker:compose} es una herramienta que permite definir y gestionar aplicaciones compuestas por múltiples contenedores mediante archivos de configuración declarativos. Facilita la orquestación de servicios, la definición de redes y volúmenes, y el despliegue conjunto de todos los componentes de una aplicación.

\subsection{Automatizaciones}

\subsubsection{GitHub Actions}
\textit{GitHub Actions}~\cite{github:actions} es una herramienta de automatización e integración continua integrada en \textit{GitHub} que permite definir flujos de trabajo (\textit{workflows}) basados en eventos del repositorio, como \textit{pushes}, \textit{pull requests} o cierres de \textit{issues}. Mediante archivos de configuración declarativos, permite automatizar tareas como la ejecución de pruebas, el análisis de código, la construcción de aplicaciones o la ejecución de herramientas externas.

\subsubsection{SonarCloud}
\textit{SonarCloud} (\textit{SonarQube Cloud})~\cite{sonarcloud:about} es una plataforma de análisis estático de código basada en la nube que permite evaluar automáticamente la calidad y seguridad del \textit{software}. Se integra en el flujo de integración continua para detectar errores, vulnerabilidades y deuda técnica, proporcionando métricas detalladas sobre la mantenibilidad y la cobertura de pruebas del proyecto.

%%%%%%%%%%%%%%%%%%%%%%%%%%%%%%%%%%%%%%%%%%%%%%%%%%%%%%%%%%%%%%%%

\section{Herramientas de desarrollo y documentación}

\subsection{Control de versiones}
\subsubsection{GitHub}
\textit{GitHub}~\cite{github:about} es una plataforma de alojamiento de repositorios que se basa en el sistema de control de versiones distribuido \textit{Git}. Permite gestionar el código fuente de un proyecto, registrar su evolución histórica y facilitar la colaboración mediante funcionalidades como repositorios remotos, gestión de ramas, control de incidencias (\textit{issues}) y solicitudes de integración (\textit{pull requests}). \textit{Git}, como sistema subyacente, permite realizar un seguimiento detallado de los cambios y trabajar de forma descentralizada.

\subsubsection{GitHub Projects}
\textit{GitHub Projects}~\cite{github:projects} es una herramienta integrada en la plataforma \textit{GitHub} orientada a la planificación y gestión del trabajo mediante tableros visuales. Permite organizar tareas a través de vistas tipo Kanban, listas o tablas, vinculando elementos como \textit{issues} y \textit{pull requests} procedentes de uno o varios repositorios.

\imagenConEtiqueta{img/apdo_4/4_tablero_kanban.png}{Tablero Kanban de \textit{GitHub Projects}}{4_tablero_kanban}

\textit{GitHub Projects} añade una capa de gestión independiente del repositorio, facilitando la priorización, el seguimiento del progreso y la visualización del estado del proyecto. Además, incorpora mecanismos de automatización (\textit{workflows}) que permiten realizar acciones automáticas en función de eventos, como actualizar el estado de una tarea al cerrarse un \textit{issue} o al cambiar su posición dentro del tablero. La herramienta también ofrece métricas y gráficos, como diagramas de progreso (\textit{burn-up}), que ayudan a analizar la evolución del trabajo a lo largo del tiempo.

\subsubsection{GitHub Flow}
\textit{GitHub Flow} es un flujo de trabajo ligero para la gestión de ramas en proyectos controlados con \textit{Git}. Se basa en una rama principal (\textit{main}) estable y en la creación de ramas independientes (\textit{/feature/x}) para el desarrollo de nuevas funcionalidades, correcciones o mejoras (en nuestro caso, se ha creado una rama por cada \textit{issue}). Una vez completado el trabajo en una rama, ésta se integra nuevamente en la rama principal mediante un proceso de revisión y fusión.

\imagenConEtiqueta{img/apdo_4/4_comparativa_githubFlow.png}{Comparativa de ramas entre \textit{GitFlow} y \textit{GitHub Flow}~\cite{medium:githubFlow}}{4_comparativa_githubFlow}

A diferencia de metodologías más complejas como \textit{Git Flow}, que define múltiples ramas permanentes (p. ej. \textit{develop}, \textit{release} o \textit{hotfix}), \textit{GitHub Flow} reduce la complejidad del proceso y resulta especialmente adecuado para proyectos de tamaño reducido o desarrollo individual, donde se prioriza la simplicidad y la trazabilidad de los cambios.

\subsubsection{\textit{Conventional Commits}}
\textit{Conventional Commits}~\cite{convComm:summary} es una convención para la redacción de mensajes de \textit{commit} que define una estructura estandarizada basada en un prefijo que indica el tipo de cambio realizado (p. ej. \textit{feat}, \textit{fix}, \textit{refactor}). Este enfoque mejora la legibilidad del historial de cambios y facilita la comprensión de la evolución del proyecto.

La estructura de un \textit{commit} según esta convención es la siguiente:
\par
{\centering \texttt{<tipo>[alcance (opcional)]: <descripción>} \par}

\subsection{Entorno de desarrollo}
\subsubsection{Visual Studio Code}
\textit{Visual Studio Code}~\cite{vscode:docs} es un editor de código fuente ligero y multiplataforma ampliamente utilizado en el desarrollo de \textit{software}. Destaca su amplio ecosistema de extensiones que permiten adaptarlo a distintos lenguajes y tecnologías. También resulta muy útil su integración nativa con \textit{Git}, que permite gestionar el control de versiones directamente desde la interfaz del editor.

Para este proyecto se ha hecho uso del 
\href{https://marketplace.visualstudio.com/items?itemName=vscjava.vscode-java-pack}{\textit{Extension Pack for Java}}, 
un conjunto de herramientas que proporciona soporte avanzado para el lenguaje. Incluye autocompletado inteligente, navegación por el código, refactorización, ejecución y depuración de aplicaciones, así como integración con herramientas de construcción y pruebas.

A pesar de que existen entornos específicos para el desarrollo con Java, como \textit{IntelliJ IDEA}~\cite{intelliJ:docs}, se optó por \textit{VS Code} debido a la familiaridad con su flujo de trabajo tras su uso continuado durante el grado. Gracias al mencionado sistema de extensiones, este editor ofrece una experiencia equiparable en potencia y funcionalidad.

\subsubsection{Postman}
\textit{Postman}~\cite{postman:product} es una herramienta de desarrollo que permite diseñar, ejecutar y analizar peticiones HTTP a servicios web, especialmente APIs REST. Facilita el envío de solicitudes con distintos métodos (\texttt{GET, POST, PUT, DELETE}, etc.), la configuración de cabeceras y cuerpos de petición, y la visualización detallada de las respuestas devueltas por el servidor. 

\textit{Postman} se utiliza habitualmente para verificar el comportamiento de los \textit{endpoints}, validar respuestas y apoyar el proceso de desarrollo y depuración de APIs. En este proyecto, se ha usado principalmente para validar la lógica de la API desarrollada con \textit{Spring Boot}, así como para testear las respuestas provenientes de las APIs nutricionales externas, facilitando así su posterior integración en el sistema.

\subsection{Documentación}
\subsubsection{LaTeX}
\LaTeX{}~\cite{latex:home} es un sistema de composición tipográfica orientado a la elaboración de documentos técnicos y científicos. Permite separar el contenido del formato, facilitando la creación de documentos estructurados y coherentes, especialmente adecuados para trabajos académicos extensos.

\subsubsection{TeXstudio}
\TeX{}studio~\cite{texstudio:home} es un entorno de desarrollo integrado para \LaTeX{} que proporciona herramientas como edición asistida, autocompletado de comandos, compilación integrada y visualización del documento resultante. Facilita la redacción y revisión de documentos complejos en \LaTeX{} de forma local.

La herramienta escogida inicialmente fue \textit{Overleaf}~\cite{overleaf:about}, con el fin de aprovechar sus funciones de colaboración y revisión remota. Sin embargo, a medida que aumentaba la extensión y complejidad de la documentación, se terminó excediendo el \textit{compile timeout} permitido en su 
\href{https://docs.overleaf.com/getting-started/free-and-premium-plans/plan-limits}{plan gratuito}. 
Es por ello que se trasladó el flujo de trabajo al entorno local mediante, manteniendo la documentación sincronizada tanto en el repositorio de \textit{GitHub} como en la plataforma de \textit{Overleaf}. De este modo, el tutor podría disponer siempre del documento más reciente para su revisión.

\subsubsection{MiKTeX}
MiK\TeX{}~\cite{miktex:home} es una distribución ligera de \LaTeX{} que incluye los paquetes y herramientas necesarios para compilar documentos. Se caracteriza por su sistema de gestión de paquetes bajo demanda, lo que permite instalar únicamente los componentes necesarios, reduciendo el tamaño de la instalación. Esta particularidad motivó la elección de MiK\TeX{} frente a otras distribuciones como \TeX{}Live~\cite{texlive:guide}, la cual realiza instalaciones completas de forma predeterminada.

\capitulo{5}{Aspectos relevantes del desarrollo del proyecto}

Este capítulo recoge los aspectos más importantes del desarrollo del proyecto. Engloba la descripción de los componentes implementados, la justificación de las decisiones tomadas y la diferenciación de las fases del desarrollo.

%%%%%%%%%%%%%%%%%%%%%%%%%%%%%%%%%%%%%%%%%%%%%%%%%%%%%%%%%%%%%%%%

\section{Inicio del proyecto}

La idea del proyecto surgió a partir de mi familiaridad con el flujo de trabajo en locales de restauración y el potencial de escalabilidad detectado, permitiendo futuras ampliaciones como  la gestión de inventario o el análisis de costes, entre otros. Asimismo, el desarrollo se planteó como una oportunidad para consolidar las competencias adquiridas durante la carrera y profundizar en tecnologías modernas de alta demanda en el mercado actual.

En la primera reunión con el tutor, se le explicaron los objetivos generales y las funcionalidades mínimas que se pretendía que abarcara la aplicación. Se acordó entonces introducir aspectos que contribuyeran a la diferenciación de la aplicación respecto de herramientas comunes de gestión. Fue ahí donde el tutor propuso tomar como referencia un Trabajo de Fin de Grado anterior, elaborado por Álvaro Manjón Vara. En dicho proyecto se desarrolló una aplicación web de nombre \textit{\textbf{NutriMenu}}~\cite{nutrimenu2024}, dirigida a los centros de restauración de la Universidad de Burgos, cuya principal funcionalidad era la creación de menús con información nutricional de cada plato, obtenida a partir de una API externa.

Se dio la circunstancia de que el \textit{stack} tecnológico sugerido al tutor coincidía en gran medida con el del TFG mencionado. Aún así, no se optó por utilizar como base dicha aplicación, sino que se partió de un desarrollo desde cero, puesto que tanto el modelo de datos como la lógica de negocio iban a verse modificados significativamente y nuevos módulos de gestión iban a ser añadidos.

%%%%%%%%%%%%%%%%%%%%%%%%%%%%%%%%%%%%%%%%%%%%%%%%%%%%%%%%%%%%%%%%

\section{Desarrollo del \textit{backend}}

\todo[inline]{intro breve con infraestructura?
	
	spring boot + mysql + docker -> sistemas distribuidos asignatura + ventajas que ofrecen


bbdd se inicializa mediante las tablas de spring boot. para los usuarios, tenemos el datainitializer %-> aunque luego vamos a tener que meter entidades (productos, alimentos, reservas) de prueba, así que queda pendiente de modificar esto

}

% MENCIONAR?
% -Métodos de consistencia -> relaciones bidireccionales
% clave compuesta derivada -> ya se menciona en anexo
% excluir campos tostring y hashcode en relaciones -> lazyinitialization, stackoverflow...

%---------------------------------------------------------------

\subsection{Funcionalidades principales y módulos}

La API REST desarrollada cubre los principales procesos de gestión de un local de restauración, agrupados en los siguientes módulos funcionales:

\begin{itemize}
	\item \textbf{Gestión de productos:} Permite la creación, edición, consulta y eliminación de productos ofertados, así como la gestión de sus ingredientes y el cálculo dinámico de su información nutricional.
	\item \textbf{Gestión de alimentos e ingredientes:} Facilita la administración de alimentos y su información nutricional, permitiendo su uso como ingredientes en productos.
	\item \textbf{Gestión de reservas de mesas:} Permite la reserva y administración de mesas, incluyendo la gestión de disponibilidad y la prevención de solapamientos.
	\item \textbf{Gestión de mesas:} Administra las mesas disponibles en el local, su capacidad y estado.
	\item \textbf{Gestión de pedidos (comandas):} Soporta la creación y seguimiento de pedidos realizados en el local, diferenciando entre pedidos de barra y de comedor mediante una jerarquía de herencia.
	\item \textbf{Gestión de usuarios y roles:} Incluye la administración de usuarios, roles y permisos, con control de acceso basado en roles y granularidad de permisos.
\end{itemize}

Cada módulo está implementado siguiendo el patrón de capas\todo{ref a apartado o anexo que lo explique o a 4.2 de técnicas y herramientas}, con entidades de dominio, repositorios, servicios y controladores, lo que facilita la mantenibilidad y la extensibilidad del sistema.

%---------------------------------------------------------------

\subsection{Modelo de datos y relaciones}

\todo[inline]{INCLUIR? -> o en anexos
	
	La persistencia se gestiona mediante repositorios que extienden JpaRepository...
	%, lo que permite aprovechar las capacidades de Spring Data JPA para la consulta y manipulación de datos.
}

El modelo de datos, específicamente en su módulo de alimentos, toma como referencia el esquema de \textit{NutriMenu}~\cite{nutrimenu2024}. Se utiliza una base de datos \textit{\textbf{MySQL}} para asegurar la consistencia relacional de la información. Al igual que en el proyecto de referencia, se emplea un \textbf{enfoque híbrido}, el cual combina el almacenamiento local de alimentos creados manualmente con la obtención de datos externos provenientes de la API de \textit{Open Food Facts}~\cite{api:openFoodFacts}.

\imagenConEtiqueta[0.6\textwidth]{img/apdo_5/5_modeloBD_nutrimenu.png}{Modelo de base de datos de \textit{NutriMenu}~\cite{nutrimenu2024}}{5_modeloBD_nutrimenu}

A pesar de compartir el planteamiento conceptual con el proyecto de referencia, se han introducido \textbf{mejoras} significativas en la arquitectura de datos y la lógica de negocio para optimizar el \textbf{rendimiento} y la \textbf{mantenibilidad} del sistema. Las modificaciones más relevantes son:

\begin{itemize}
	\item \textbf{Simplificación mediante objetos embebidos:} A diferencia de \textit{NutriMenu}, que gestionaba la información nutricional (vitaminas, minerales, etc.) como entidades independientes, en este trabajo se han implementado como objetos embebidos (\texttt{\char`@{}Embeddable}). Esto elimina la necesidad de tablas adicionales y relaciones complejas (\textit{joins}), permitiendo que toda la información nutricional se persista en una única tabla de alimentos, lo que mejora la eficiencia de las operaciones en la base de datos.
	\item \textbf{Cálculo dinámico de información nutricional y consistencia:} Se ha sustituido la actualización manual de valores nutricionales por un sistema de cálculo dinámico. Mientras que en el proyecto anterior los cálculos eran verbosos y se duplicaban en la base de datos, este modelo utiliza el principio \textit{``\textbf{Single Source of Truth}''}~\cite{medium:singleSourceTruth}, recalculando los valores en tiempo real a partir del alimento original. Esto asegura que cualquier cambio en un ingrediente se refleje automáticamente en todos los productos asociados.
\end{itemize}

Para una descripción técnica más detallada sobre la implementación de estas entidades y la lógica de negocio aplicada, se recomienda consultar los \textit{pull requests} 
\href{https://github.com/aperezolmos/tfg-aperezolmos/pull/51}{\#51} y 
\href{https://github.com/aperezolmos/tfg-aperezolmos/pull/52}{\#52} 
en el repositorio del proyecto.

\imagenConEtiqueta[0.6\textwidth]{img/apdo_5/5_modeloEER_Completo.png}{Modelo de base de datos actual}{5_modeloEER_Completo}

\todo[inline]{Métodos de consistencia bidireccional y campos de auditoría -> aquí o en el apéndice C??}

%---------------------------------------------------------------

\subsection{Lógica de negocio y servicios}

La lógica de negocio se encapsula en servicios específicos para cada módulo, que se encargan de validar las operaciones, gestionar las transacciones\todo{mencionar anotación Transactional? -> quizás mejor en anexo diseño procedimental} y aplicar las reglas del dominio. Por ejemplo:

\begin{itemize}
	\item El servicio de \textbf{productos} calcula la información nutricional total de cada producto en función de sus ingredientes, utilizando métodos de dominio y operaciones sobre objetos de tipo \texttt{NutritionInfo}.
	\item El servicio de \textbf{reservas} valida la disponibilidad de mesas y previene solapamientos mediante la comprobación de conflictos de horario antes de confirmar una reserva.
	\item Los servicios de \textbf{pedidos} gestionan la creación y actualización de comandas, diferenciando entre pedidos de barra y comedor mediante una jerarquía de herencia (\texttt{Order}, \texttt{BarOrder}, \texttt{DiningOrder}) y un patrón de factoría\todo{ref} para los \textit{mappers} (\texttt{OrderMapperFactory})\todo{explicar en anexos}.
\end{itemize}

La separación de responsabilidades\todo{ref principio SRP?} 
%entre servicios y controladores 
permite mantener la lógica de negocio aislada de los puntos de entrada de la API, facilitando la reutilización y el testeo de los componentes. Este diseño ha favorecido la implementación de \textit{\textbf{thin controllers}}\todo{ref?}, esto es, controladores que se mantienen ligeros al delegar toda la complejidad y las reglas de dominio en los servicios correspondientes.

%---------------------------------------------------------------

\subsection{Comunicación entre capas: DTOs y \textit{mappers}}

La comunicación entre las distintas capas de la aplicación se realiza mediante objetos de transferencia de datos, diferenciando entre DTOs de \textbf{petición} (\textit{RequestDTO}) y de \textbf{respuesta} (\textit{ResponseDTO}). Esta distinción permite adaptar los datos expuestos por la API a las necesidades de cada operación, facilitando la validación de los datos recibidos (ver apartado ``\textit{\nameref{subsec:5.2_excepVal}}'') y ocultando detalles internos o sensibles. Por ejemplo, la contraseña de un usuario, aunque se almacena encriptada, no se proporciona en el DTO de respuesta.

La conversión entre entidades y DTOs se realiza mediante \textit{mappers} implementados con \textit{MapStruct}, lo que permite definir reglas de mapeo declarativas y reducir la cantidad de código repetitivo. En casos complejos, como la jerarquía de pedidos, se emplea una factoría de \textit{mappers} para aislar la lógica de conversión y facilitar la extensión a nuevos tipos de pedidos.\todo{incluir esta última línea?}

\missingfigure{diagrama de comunicación entre capas, indicando dónde se usa el dto y dónde la entidad}

%---------------------------------------------------------------

\subsection{Seguridad y control de acceso}

La seguridad y el control de acceso en el \textit{backend} se gestionan mediante \textit{Spring Security}, implementando un sistema de autenticación basado en sesiones HTTP y \textbf{autorización granular por roles y permisos} 
(ver \textit{pull request} \href{https://github.com/aperezolmos/tfg-aperezolmos/pull/81}{\#81}). 
Los \textit{endpoints} de la API están protegidos mediante anotaciones como \texttt{\char`@{}PreAuthorize}, que verifican los permisos necesarios para cada operación, garantizando que solo los usuarios autorizados puedan acceder a los recursos sensibles.

La sesión de usuario autenticado se mantiene en el \textit{backend} mediante el componente \texttt{SessionManager}, que almacena el contexto de seguridad en la sesión HTTP y gestiona el ciclo de vida de la autenticación. Para la representación de usuarios en el sistema de seguridad, se utiliza la clase \texttt{CustomUserDetails}, que adapta la entidad \texttt{User} y expone sus roles y permisos como \textit{authorities} de \textit{Spring Security}, permitiendo así un control de acceso detallado a los recursos.


%más breve
%La seguridad de la aplicación se gestiona mediante Spring Security, implementando control de acceso basado en roles y permisos. Los endpoints de la API están protegidos mediante anotaciones de autorización (\texttt{\char`@{}PreAuthorize}), y la autenticación se realiza mediante sesiones gestionadas en el backend. El sistema de roles y permisos permite una granularidad suficiente para cubrir los distintos perfiles de usuario del local.

%---------------------------------------------------------------

\subsection{Manejo de excepciones y validaciones}
\label{subsec:5.2_excepVal}

El \textit{backend} implementa un \textbf{sistema centralizado de manejo de excepciones} mediante el componente \texttt{GlobalExceptionHandler}, que captura y gestiona los errores más frecuentes (entidades no encontradas, violaciones de integridad, errores de validación, etc.). Los errores se devuelven en un formato estructurado que facilita su tratamiento en el \textit{frontend}.

\missingfigure{Captura del formato de ErrorResponse}

Las validaciones de los datos de entrada se realizan tanto a nivel de DTOs, utilizando anotaciones estándar de \textit{Java Validation} (\texttt{\char`@{}NotNull}, \texttt{\char`@{}Size}, \texttt{\char`@{}Min}, etc.), como a nivel de lógica de negocio en los servicios. 

%%%%%%%%%%%%%%%%%%%%%%%%%%%%%%%%%%%%%%%%%%%%%%%%%%%%%%%%%%%%%%%%

\section{Desarrollo del \textit{frontend}}

...

\subsection{Internacionalización}

...

%%%%%%%%%%%%%%%%%%%%%%%%%%%%%%%%%%%%%%%%%%%%%%%%%%%%%%%%%%%%%%%%

\section{Consumo de la API de datos nutricionales}
%Otro título: "Suministro de información nutricional mediante API"

La elección de una API que proporcionase información de alimentos y datos nutricionales se llevó a cabo mediante una comparativa que se encuentra detallada en el \textit{issue} \href{https://github.com/aperezolmos/tfg-aperezolmos/issues/10}{\#10}. La API seleccionada fue \textit{\textbf{Nutritionix}}~\cite{api:nutritionix}, la cual también fue elegida en el proyecto \textit{NutriMenu}~\cite{nutrimenu2024}.

Sin embargo, a 19 de noviembre de 2025, recibí una notificación oficial vía correo electrónico por parte de \textit{Nutritionix}, en la cual se comunicaba que el plan de acceso gratuito dejaría de estar operativo~\cite{reddit:nutritionixFreePlan}, invalidando así su uso para el proyecto.

Se optó entonces por la segunda opción más completa de la comparativa: \textbf{\textit{Open Food Facts} (OFF)}~\cite{api:openFoodFacts}.
\todo[inline]{Qué ventajas tiene respecto a ntx (alérgenos, abierto) y desventajas (productos comunes) -> breve (1-2 líneas)}

\todo[inline]{Después aquí explicar brevemente los endpoints utilizados o los datos extraídos de la api. -> o hacer esto en apartado 4. Técnicas y herramientas?}

\missingfigure{Búsqueda de un alimento mediante la API OFF}

%%%%%%%%%%%%%%%%%%%%%%%%%%%%%%%%%%%%%%%%%%%%%%%%%%%%%%%%%%%%%%%%

\section{TO DO: otras secciones}

Despliegue, Testing, Automatizaciones...
\capitulo{6}{Trabajos relacionados}

% Este apartado sería parecido a un estado del arte de una tesis o tesina. En un trabajo final grado no parece obligada su presencia, aunque se puede dejar a juicio del tutor el incluir un pequeño resumen comentado de los trabajos y proyectos ya realizados en el campo del proyecto en curso.


\section{NutriMenu}

La aplicación \textit{\textbf{NutriMenu}}~\cite{nutrimenu2024} fue desarrollada por el alumno Álvaro Manjón Vara como Trabajo de Fin de Grado [...]\todo{completar}

\capitulo{7}{Conclusiones y Líneas de trabajo futuras}

\section{Conclusiones}

El desarrollo de este Trabajo de Fin de Grado me ha permitido abordar de forma práctica el ciclo completo de desarrollo de una aplicación web, desde el análisis y diseño hasta la implementación, despliegue y documentación. Considero que los \textbf{objetivos} planteados al inicio del proyecto se han cumplido satisfactoriamente, ya que se ha obtenido una aplicación funcional y coherente para la gestión de locales de restauración, capaz de centralizar procesos clave como la gestión de productos, pedidos, reservas, mesas y usuarios. Además, la arquitectura diseñada sienta una \textbf{base sólida} para futuras ampliaciones y mejoras.

A nivel técnico, el proyecto ha supuesto una importante oportunidad de \textbf{aprendizaje}, permitiéndome afianzar conceptos generales de ingeniería del \textit{software} y del ciclo de vida del desarrollo adquiridos a lo largo del grado. Destaca especialmente el desarrollo del \textit{frontend} con \textit{React}, tecnología que no había utilizado previamente y que me ha permitido adquirir competencias en el desarrollo de interfaces web modernas. En el \textit{backend}, se ha dado una gran importancia al diseño del modelo de datos y de la lógica de negocio, priorizando la mantenibilidad y la escalabilidad de los componentes. Además, el desarrollo del proyecto ha requerido una planificación cuidadosa del trabajo, fomentando una correcta distribución y priorización de tareas a lo largo del tiempo. 

En este contexto, la implementación de un sistema flexible de roles y \textbf{permisos} aporta un valor significativo a la aplicación, ya que permite adaptarse a \textbf{escenarios reales} donde los puestos de trabajo y sus responsabilidades son cambiantes y específicos, ofreciendo al usuario la capacidad de definir y gestionar sus propios perfiles de acceso según sus necesidades.

Otro aspecto diferencial de la aplicación es la \textbf{integración de la API} de \textit{Open Food Facts}, la cual ha permitido enriquecer los productos con información nutricional, alérgenos y métricas como \textit{Nutri-Score} y NOVA. Este aspecto, unido a mi interés personal y experiencia previa en el sector de la restauración, ha hecho que el desarrollo del proyecto resultara especialmente motivador y satisfactorio, consolidando tanto conocimientos técnicos como habilidades de planificación y organización del trabajo.

\section{Líneas de trabajo futuras}

Como se ha mencionado anteriormente, \textit{Orderly} nace con una arquitectura diseñada para la \textbf{escalabilidad}. Aunque la versión actual cumple con los objetivos de gestión básicos, el sistema se ha concebido como una \textbf{base} sobre la cual se pueden implementar múltiples módulos adicionales. A continuación, se detallarán las principales vías de expansión propuestas.

\subsubsection{Evolución del modelo y gestión de productos}
\begin{itemize}
	\item \textbf{Diferenciación de tipos de producto:} clasificación detallada de ítems (entrantes, platos principales, bebidas, postres) para mejorar la organización de la carta.
	\item \textbf{Venta por peso y unidades de medida:} soporte para precios por kilogramo o litro, permitiendo una gestión más precisa de productos a granel.
	\item \textbf{Escandallo de costes:} implementación de una herramienta para calcular el coste de producción de cada plato a partir de sus ingredientes y determinar márgenes de beneficio óptimos.
\end{itemize}

\subsubsection{Módulo visual de sala y reservas}
\begin{itemize}
	\item \textbf{Mapa dinámico del comedor:} sustitución de los listados de mesas por una interfaz gráfica interactiva que represente la disposición física del local.
	\vspace{2em}
	\item \textbf{Gestión avanzada de reservas:}
	\begin{itemize}
		\item Asignación de reservas directamente sobre el mapa visual.
		\item Historial de ocupación por mesa y tramos horarios.
		\item Sistema de avisos y recordatorios automáticos para clientes.
	\end{itemize}
\end{itemize}

\subsubsection{Experiencia del cliente y salud}
\begin{itemize}
	\item \textbf{Carta digital interactiva:} expansión de la interfaz para uso de clientes, permitiendo la consulta de platos y su información nutricional en tiempo real.
	\item \textbf{Generador de menús dinámicos:} creación de menús configurables (ej. menú del día, fin de semana) con reportes nutricionales agregados.
	\item \textbf{Recomendaciones saludables:} implementación de un motor de sugerencias basado en los perfiles nutricionales obtenidos de la API externa.
\end{itemize}

\subsubsection{Optimización operativa y analítica}
\begin{itemize}
	\item \textbf{Panel de visualización para cocina:} interfaz específica para el personal de cocina que permita gestionar el estado de los pedidos (en preparación, listo para servir) mediante actualizaciones en tiempo real.
	\item \textbf{Estadísticas:} creación de un \textit{dashboard} analítico para monitorizar los productos más vendidos, horas de mayor afluencia y rendimiento económico.
	\item \textbf{Gestión de inventario automática:} descuento automático de existencias tras cada pedido e integración de alertas de \textit{stock} bajo.
\end{itemize}

\subsubsection{Mejoras técnicas y de accesibilidad}
\begin{itemize}
	\item \textbf{Seguridad y recuperación:} implementación de un flujo de recuperación de contraseñas mediante envío de credenciales temporales al correo electrónico.
	\item \textbf{Integración con servicios de terceros:} exploración de conexiones con plataformas de \textit{delivery} e integración de pasarelas de pago \textit{online}.
	\item \textbf{Calidad de software:} ampliación de la cobertura de pruebas unitarias e integración de pruebas de extremo a extremo (E2E) para el \textit{frontend}.
\end{itemize}

\subsubsection{Eficiencia técnica y rendimiento del sistema}
\begin{itemize}
	\item \textbf{Estrategias de almacenamiento en caché (\textit{Caching}):}
	\begin{itemize}
		\item \textbf{Caché de servicios externos:} implementar una capa de caché para almacenar los resultados de la API nutricional, reduciendo el tiempo de respuesta y el consumo de cuota de la API externa (en caso de integrar una nueva API que tenga límite de peticiones).
		\item \textbf{Caché en el cliente}: optimizar la comunicación entre \textit{React} y \textit{Spring Boot}, evitando peticiones redundantes al \textit{backend} cuando los datos no han variado.
	\end{itemize}
	\item \textbf{Sistema de refresco automático de información nutricional:} establecer un mecanismo basado en un tiempo de vida (\textit{Time-To-Live} o \textit{TTL}) para los datos de los alimentos. Una vez superado dicho umbral, el sistema realizaría una petición automática de actualización a la API para garantizar que la información mostrada sea siempre vigente y precisa.
\end{itemize}

En definitiva, \textit{Orderly} se posiciona como una solución versátil que puede evolucionar desde una herramienta interna de gestión hasta convertirse en una plataforma integral que conecte a trabajadores y clientes bajo un ecosistema digital común.



\bibliographystyle{plain}
\bibliography{bibliografia}

\end{document}
